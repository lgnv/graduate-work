\Defines % Необходимые определения. Вряд ли понадобться
В настоящей работе применяются следующие термины
\begin{description}
\item[Ulearn] платформа с интерактивными онлайн-курсами по программированию
\item[Codeforces] проект, объединяющий людей, которые интересуются и участвуют в соревнованиях по программированию. Является как социальной сетью, посвященной программированию и соревнованиям так и площадкой, где регулярно проводятся соревнования
\item[Timus Online Judge] крупнейший в России архив задач по программированию с автоматической проверяющей системой.
\item[Формат задач Codeforces] принятый стандарт внутреннего описания олимпиадных задач. Распространяется также на Timus Online Judge
\item[Polygon] платформа для создания задач по программированию
\item[Docker] программное обеспечение с открытым исходным кодом, применяемое для разработки, тестирования, доставки и запуска веб-приложений в средах с поддержкой контейнеризации
\item[Чеккер] программа, выполняющая задачу проверки корректности ответа

\item[Don’t repeat yourself] принцип разработки программного обеспечения, нацеленный на снижение повторения информации различного рода, особенно в системах со множеством слоёв абстрагирования.
\item[Single-responsibility principle] принцип ООП, обозначающий, что каждый объект должен иметь одну ответственность и эта ответственность должна быть полностью инкапсулирована в класс. Все его поведения должны быть направлены исключительно на обеспечение этой ответственности.
\end{description}
%%% Local Variables:
%%% mode: latex
%%% TeX-master: "rpz"
%%% End:
