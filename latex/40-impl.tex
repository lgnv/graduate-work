\definecolor{gray}{rgb}{0.4,0.4,0.4}
\definecolor{darkblue}{rgb}{0.0,0.0,0.6}
\definecolor{cyan}{rgb}{0.0,0.6,0.6}
\lstloadlanguages{C,C++,csh,Java}

\definecolor{red}{rgb}{0.6,0,0} 
\definecolor{blue}{rgb}{0,0,0.6}
\definecolor{green}{rgb}{0,0.8,0}
\definecolor{cyan}{rgb}{0.0,0.6,0.6}

\lstset{
  basicstyle=\ttfamily,
  columns=fullflexible,
  showstringspaces=false,
  commentstyle=\color{gray}\upshape
}

\lstdefinelanguage{XML}
{
  morestring=[b]",
  morestring=[s]{>}{<},
  morecomment=[s]{<?}{?>},
  stringstyle=\color{black},
  identifierstyle=\color{darkblue},
  keywordstyle=\color{cyan},
  morekeywords={xmlns,version,type}% list your attributes here
}


\section{Техническая реализация}
\label{cha:impl}

В этом разделе описана техническая часть работы, реализованная в рамках спроектированных в прошлом пункте идей.


\subsection{Xsd-схема блока задачи}

Как можно увидеть в Листинге 1.10 типизация тега slide.polygon не наследуется ни от каких других типов, а значит автору будет подсказываться только то, что действительно нужно.

\begin{lstlisting}[caption={Реализация xsd-схемы тега slide.polygon}, language={XML}]
<xs:element name="slide.polygon" type="PolygonSlide" />
<xs:complexType name="PolygonSlide">
	<xs:all>
		<xs:element name="scoring"type="ExerciseScoring" minOccurs="0"/>
		<xs:element name="polygonPath"type="xs:string"/>
		<xs:element name="markdown"type="MarkdownBlock" minOccurs="0"/>
		<xs:element name="exercise.polygon"type="PolygonExercise" />
	</xs:all>
	<xs:attribute name="id" type="NonEmptyString"use="required" />
	<xs:attribute name="hide" type="xs:boolean" />
</xs:complexType>
\end{lstlisting}

\subsection{Обработка конфигурации слайда}

За обработку конфигурации слайдов с задачами отвечает сущность ExerciseSlide. Если задачи типа UniversalExercise, то внутри ExerciseSlide находится блок  поэтому необходимо наш конфигуратор PolygonExerciseSlide отнаследовать от него.


Первая задача, которую нужно решить для генерации слайда — получить доступ к данным о задаче. Для этого вынесем свойство нашей сущности путь до директории с этой информацией.

На этапе валидации слайда необходимо знать, что автор курса не забыл про это поле и указал его. Поэтому оно проверяется явно. Это представлено на листинге 1.11.

\begin{lstlisting}[caption={Проверка полей}]
    [XmlElement("polygonPath")]
    public string PolygonPath { get; set; }
    public override void Validate(SlideLoadingContext context)
		{
			if (string.IsNullOrEmpty(PolygonPath))
				throw new CourseLoadingException("polygonPath not found in slide.polygon");
			base.Validate(context);
		}
\end{lstlisting}

Полный текст задачи генерируется автоматически благодаря вложенной в архив задачи информации. В рассмотренном листинге 1.1 показано, что существует файлы problem.html и problem-statements.css, которые находятся по пути /statements/.html/russian/problem.html и /statements/.html/russian/problem-statements.html соответственно. Они определяют визуальное представление задачи, которое состоит из легенды, примеров входных и выходных данных, информации об ограничениях задачи.
Эти файлы используются в нашей обработке конфигурации: файлы читаются и их содержимое подставляется в html-код, выводящийся пользователю.

Оставлена возможность для автора курса на своё усмотрение переопределить выводящуюся пользователю информацию. Для этого ему необходимо внутри слайда добавить тег markdown с необходимыми данными и он будет рассматриваться приемущественнее, чем текст в архиве задачи.

Так же если в архиве задачи есть визуальное представление в виде pdf, то пользователю дается ссылка на этот файл.\\

Эта логика представлена в листинге 1.12.
\begin{lstlisting}[caption={Формирование блоков}]
private IEnumerable<SlideBlock> GetBlocksProblem(
        string statementsPath, 
        string courseId, 
        Guid slideId)
		{
			var markdownBlock = Blocks.FirstOrDefault(block => block is MarkdownBlock);
			if (markdownBlock != null)
			{
				yield return markdownBlock;
			}
			else if(Directory.Exists(Path.Combine(statementsPath, ".html")))
			{
				var htmlDirectoryPath = Path.Combine(statementsPath, ".html", "russian");
				var htmlData = File.ReadAllText(
				Path.Combine(htmlDirectoryPath, "problem.html"));
				yield return RenderFromHtml(htmlData);
			}

			var pdfLink = PolygonPath + "/statements/.pdf/russian/problem.pdf";
			yield return new MarkdownBlock($"[Download problem conditions in PDF format]({pdfLink})");
		}
\end{lstlisting}

Далее описана непосредственно обработка конфигурации в методе BuildUp. Рассмотрим листинг 1.13. 

\begin{lstlisting}[caption={Обработка конфигурации в PolygonExerciseSlide}]
public override void BuildUp(SlideLoadingContext context)
{
	var statementsPath = Path.Combine(context.Unit.Directory.FullName, PolygonPath, "statements");
			
	Blocks = GetBlocksProblem(statementsPath, context.CourseId, Id)
			.Concat(Blocks.Where(block => !(block is MarkdownBlock)))
			.ToArray();
			
	var problem = GetProblem(
		Path.Combine(context.Unit.Directory.FullName, PolygonPath, "problem.xml"), 
		context.CourseSettings.DefaultLanguage
	);

	var polygonExercise = Blocks.Single(block => block is PolygonExerciseBlock) as PolygonExerciseBlock;
			
	polygonExercise!.ExerciseDirPath = Path.Combine(PolygonPath);
	polygonExercise.TimeLimitPerTest = problem.TimeLimit;
	polygonExercise.TimeLimit = (int)Math.Ceiling(problem.TimeLimit * problem.TestCount);
	polygonExercise.UserCodeFilePath = problem.PathAuthorSolution;
	polygonExercise.Language = LanguageHelpers.GuessByExtension(new FileInfo(polygonExercise.UserCodeFilePath));
	polygonExercise.DefaultLanguage = context.CourseSettings.DefaultLanguage;
	polygonExercise.RunCommand = $"python3.8 main.py {polygonExercise.Language} {polygonExercise.TimeLimitPerTest} {polygonExercise.UserCodeFilePath.Split('/', '\\')[1]}";
			
	Title = problem.Title;
			
	PrepareSolution(
    	Path.Combine(context.Unit.Directory.FullName, PolygonPath, polygonExercise.UserCodeFilePath)
	);

	base.BuildUp(context);
}

\end{lstlisting}

В строке 3 формируется путь до каталога statements.

В строках 5-7, используя уже рассмотренный метод, генерируются блоки представления задачи.

В строках 9-12 вызывается метод, который из файла problem.xml получает информацию о задаче: ограничение по времени на тест, количество тестов, название задачи, путь до авторского решения. 

В строке 14 из множества блоков данного слайда получаю конфигуратор блока PolygonExercise и в строках 16-22 происходит его инициализация:
\begin{enumerate}
    \item пути до директории с данными задачи;
    \item ограничении времени на тест и на выполнение всех тестов;
    \item пути до авторского решения;
    \item языка авторского решения и языка по-умолчанию(язык, который будет предложен пользователю курса в первую очередь);
    \item команду для запуска проверяющей программы внутри docker-контейнера.
\end{enumerate}

В строках 26-28 происходит вызов метода PrepareSolution, задача которого обернуть авторского решение в region, как это представлено в листинге 1.14. Внутрь этого блока будет подставляться решение студента при отправке.
\begin{lstlisting}[caption={Оборачивание кода в region}]
\\region Task
    Code
\\endregion Task
\end{lstlisting}

Завершает метод вызов base.BuildUp, который конфигурируют всю остальную логику связанную с формированием слайдов.

\subsection{Обработка конфигурации блока}

За обработку конфигурации блока, как уже было сказано в прошлом пункте, отвечает класс PolygonExerciseBlock, который является потомком UniversalExerciseBlock.

В данном классе расположена информация о языках, проверка на которых доступна студенту. Отрывок инициализации представлен в листинге 1.15.

\begin{lstlisting}[caption={Отрывок инициализации LanguagesInfo}]
public static Dictionary<Language, LanguageLaunchInfo> LanguagesInfo = new Dictionary<Language, LanguageLaunchInfo>
		{
			[Common.Language.Java] = new LanguageLaunchInfo
			{
				Compiler = "Java 14",
				CompileCommand = "javac {source}",
				RunCommand = "java {compilation-result-file}"
			},
			...
		}
\end{lstlisting}

Наибольший интерес в данном классе представляют метод создания RunnerSubmission — сущности содержащую в себе информацию об отправке студентом кода на проверку и способе его запуска. Единственно, что входит в ответственность PolygonExerciseBlock в этой задаче — формирование команды запуска. Всё остальную логику выполняет UniversalExerciseBlock. Данная логика описана в листинге 1.16.

\begin{lstlisting}[caption={Формирование RunnerSubmission}]
public RunnerSubmission CreateSubmission(string submissionId, string code, Language language)
{
	var submission = 
	    base.CreateSubmission(submissionId, code);
	if (!(submission is CommandRunnerSubmission commandRunnerSubmission))
		return submission;
	
	commandRunnerSubmission.RunCommand = RunCommandWithArguments(language);
	
	return commandRunnerSubmission;
}

private string RunCommandWithArguments(Language language)
{
	return $"python3.8 main.py {language} {TimeLimitPerTest} {UserCodeFilePath.Split('/', '\\')[1]}";
}
\end{lstlisting}

Внутри Docker-контейнера будет выполняться команда, которая находится в свойстве RunCommand. Она запустит python-код и передаст ему в аргументы язык, на котором написано решение, ограничение по времени на тест, а так же путь до файла с решением.

\subsection{Проверка решения студента}

После отправки решения студентом, запускается docker-контейнер в который подкладывается отправленное решение и начинает выполнение проверяющая программа. 

Рассмотрим конфигурацию из Dockerfile, описанную в листинге 1.17.

\begin{lstlisting}[caption={Конфигурация docker-образа}]
FROM ubuntu:20.04

ENV NVM_DIR /usr/local/nvm
ENV NODE_VERSION 14.15.3
ARG DEBIAN_FRONTEND=noninteractive
ENV TZ=Europe/Ekaterinburg

RUN apt-get update 
&& apt-get -y upgrade
&& apt-get install -y wget 
&& apt-get -y install curl \
# Python 3.8
&& apt-get -y install python3.8 \ 
# C, C++
&& apt-get -y install build-essential \
# C\#
&& wget https://packages.microsoft.com/config
    /ubuntu/20.10/packages-microsoft-prod.deb -O packages-microsoft-prod.deb \
&& dpkg -i packages-microsoft-prod.deb \
&& apt-get update && apt-get install -y apt-transport-https && apt-get install -y dotnet-sdk-5.0 \
&& apt-get install -y apt-transport-https && apt-get install -y dotnet-runtime-5.0
# Java
RUN apt-get install -y openjdk-14-jdk \
# JavaScript
&& curl --silent -o- https://raw.githubusercontent.com
    /creationix/nvm/v0.31.2/install.sh | bash \
&& . $NVM_DIR/nvm.sh \
&& nvm install $NODE_VERSION \
&& nvm alias default $NODE_VERSION \
&& nvm use default

ENV NODE_PATH $NVM_DIR/v$NODE_VERSION/lib/node_modules
ENV PATH $NVM_DIR/versions/node/v$NODE_VERSION/bin:$PATH
ENV NODE_REPL_HISTORY ''

COPY ./app/ /app/

WORKDIR app

RUN useradd student && chmod 700 /app/tests
\end{lstlisting}

Комментариями в коде помечены языки, которые становятся доступны после выполнения этих команд. 

Обратим внимание на строку 39, в которой создается новый пользователь и устанавливаются права доступа на директорию с тестами. Таким образом обеспечивается невозможность получения тестовых данных из решения.

Далее рассмотрим механизм проверки решения. 

Первоочередная задача, которая возникает в процессе выполнения проверя — определить язык решения и команды для компиляции и выполнения исходного кода.

Для инкапсуляции данных об этих командах был создан контракт ISourceCodeRunInfo. Рассмотрим листинге 1.18.
\begin{lstlisting}[caption={Контракт ISourceCodeRunInfo}]
class ISourceCodeRunInfo:
    __metaclass__ = ABCMeta
    @abstractmethod
    def file_extension(self): pass
    @abstractmethod
    def format_build_command(self, 
        code_filename: str, 
        result_filename: str) -> str: pass
    @abstractmethod
    def need_build(self) -> bool: pass
    @abstractmethod
    def format_run_command(self, filename: str) -> str: pass
\end{lstlisting}

Метод \texttt{file\_extension} возвращает расширение файла с исходным кодом. \\
Метод \texttt{format\_build\_command} выполняет задачу формирования команды для компилирования исходного кода. В случае если исходный код некомпилируемый, например, исходных код на языке Python, то возвращается None.\\
Метод \texttt{format\_run\_command} возвращает команду для запуска решения.

Приведу пример реализации данной сущность для языка C++ в листинге 1.19.

\begin{lstlisting}[caption={Реализация ISourceCodeRunInfo для языка C++}]
class CppRunInfo(ISourceCodeRunInfo):
    def file_extension(self):
        return '.cpp'
    def format_run_command(self, filename: str) -> str:
        return f'./{filename}'
    def need_build(self) -> bool:
        return True
    def format_build_command(self, code_filename: str, result_filename: str) -> str:
        return f'g++ -o {result_filename} -O2 {code_filename}'
\end{lstlisting}

Для поиска нужной реализации данного контракта по названию языка была написана функция \texttt{get\_run\_info\_by\_language\_name}, она представлена в листинге 1.20.
\begin{lstlisting}[caption={Функция поиска по названию языка реализации ISourceCodeRunInfo}]
def get_run_info_by_language_name(language_name: str) -> ISourceCodeRunInfo:
    if language_name == 'cpp':
        return CppRunInfo()
    elif language_name == 'python3':
        return PythonRunInfo()
    ...
\end{lstlisting}

В листинге 1.21 представленн класс TaskCodeRunner, который реализует логику сборку, запуск кода и запуск теста, используя взаимодействие с процессами. 

\begin{lstlisting}[caption={Сигнатура класса TaskCodeRunner}]
class TaskCodeRunner:
    def build(self, code_filename: str, result_filename: str): pass
    def run(self, test_file: str): pass
    def run_test(self, code_filename: str, test_file: str): pass
\end{lstlisting}

Проверка решения студента осуществляется по следующему алгоритму:
\begin{enumerate}
    \item сборка исходного кода чеккера check.cpp, который мы получили из директории задачи;
    \item предобработка решения студента: удаления строк определения региона; в случае с java кодом переименование названия файла в соответствие с названием класса внутри файла;
    \item получение ISourceCodeRunInfo по названию языка полученному в аргументах программы;
    \item компиляция исходного кода студента, если это требуется
    \item для каждого теста
    \begin{enumerate}
        \item запуск программы студента на этом тесте
        \item сравнение результата выполнения с авторским решением используя чеккер
    \end{enumerate}
    \item обработка ошибок, при возникновении
\end{enumerate}

Результатом выполнения проверяющей программы является json, структура которого продемонстрирована в листинге 1.22.

\begin{lstlisting}[caption={RunningResult}]
{
    "Verdict": "Ok" | "WrongAnswer" | "TimeLimit" | ...
    "TestResultInfo": {
            "TestNumber": "1" | "2" | ... ,
            "Input": // test data,
            "CorrectOutput": // correct test output ,
            "StudentOutput": // student output
        }
}
\end{lstlisting}

\subsection{Логирование}

Помимо этого, у авторов курсов могут возникнуть вопросы почему решение студента не проходит определенные тесты или подозрение в том, что система работает неправильно. Поэтому было реализовано логирование жизненного цикла проверяющей системы с выводом в интерфейс, но только для администраторов данного курса. 

\subsection{Добавление нового языка}
В дальнейшем у авторов может возникнуть желание добавить новый язык программирования, чтобы у студентов был еще более разнообразный выбор. Для встраивания еще одного языка следует выполнить следующие шаги для этого языка: 

\begin{enumerate}
    \item прописать команды установки окружния в dockerfile;
    \item реализовать сущность ISourceCodeRunInfo;
    \item прописать условие для обнаружения ISourceCodeRunInfo по названию языка в методе \texttt{get\_run\_info\_by\_language\_name};
    \item в классе PolygonExerciseBlock прописать информацию о языке в поле LanguagesInfo.
\end{enumerate}

\subsection{Тестирование}

Логика, реализованная в рамках данной работы, была протестирована несколькими методами. 

\begin{enumerate}
    \item логика работы docker-образа была проверена unit-тестом, который запускает его, отправляет туда тестовую задачу с авторским решением и проверяет, что результатом выполнения стал статус Ok;
    \item курс для тестирования системы Ulearn был дополнен несколькими слайдами, где заложена возможность ручного тестирования данной логики.
\end{enumerate}




% В дfileанном разделе описано изготовление и требование всячины. Кстати,
% в Latex нужно эскейпить подчёркивание (писать <<\verb|some\_function|>> для \Code{some\_function}).

% \ifPDFTeX
% Для вставки кода есть пакет \Code{listings}. К сожалению, пакет \Code{listings} всё ещё
% работает криво при появлении в листинге русских букв и кодировке исходников utf-8.
% В данном примере он (увы) на лету конвертируется в koi-8 в ходе сборки pdf.

% Есть альтернатива \Code{listingsutf8}, однако она работает лишь с
% \Code{\textbackslash{}lstinputlisting}, но не с окружением \Code{\textbackslash{}lstlisting}

% Вот так можно вставлять псевдокод (питоноподобный язык определен в \Code{listings.inc.tex}):

% \begin{lstlisting}[style=pseudocode,caption={Алгоритм оценки дипломных работ}]
% def EvaluateDiplomas():
%     for each student in Masters:
%         student.Mark := 5
%     for each student in Engineers:
%         if Good(student):
%             student.Mark := 5
%         else:
%             student.Mark := 4
% \end{lstlisting}

% Еще в шаблоне определен псевдоязык для BNF:

% \begin{lstlisting}[style=grammar,basicstyle=\small,caption={Грамматика}]
%   ifstmt -> "if" "(" expression ")" stmt |
%             "if" "(" expression ")" stmt1 "else" stmt2
%   number -> digit digit*
% \end{lstlisting}

% В листинге~\ref{lst:sample01} работают русские буквы. Сильная магия. Однако, работает
% только во включаемых файлах, прямо в \TeX{} нельзя.

% % Обратите внимание, что включается не ../src/..., а inc/src/...
% % В Makefile есть соответствующее правило для inc/src/*,
% % которое копирует исходные файлы из ../src и конвертирует из UTF-8 в KOI8-R.
% % Кстати, поэтому использовать можно только русские буквы и ASCII,
% % весь остальной UTF-8 вроде CJK и египетских иероглифов -- нельзя.

% \lstinputlisting[language=C,caption=Пример (\Code{test.c}),label=lst:sample01]{listings/test.c}

% \else

% Для вставки кода есть пакет \texttt{minted}. Он хорош всем кроме: необходимости Python (есть во всех нормальных (нет, Windows, я не про тебя) ОС) и Pygments и того, что нормально работает лишь в \XeLaTeX.

% Можно пользоваться расширенным BFN:

% \begin{listing}[H]
% \begin{ebnfcode}
%  letter = "A" | "B" | "C" | "D" | "E" | "F" | "G"
%       | "H" | "I" | "J" | "K" | "L" | "M" | "N"
%       | "O" | "P" | "Q" | "R" | "S" | "T" | "U"
%       | "V" | "W" | "X" | "Y" | "Z" ;
% digit = "0" | "1" | "2" | "3" | "4" | "5" | "6" | "7" | "8" | "9" ;
% symbol = "[" | "]" | "{" | "}" | "(" | ")" | "<" | ">"
%       | "'" | '"' | "=" | "|" | "." | "," | ";" ;
% character = letter | digit | symbol | "_" ;
 
% identifier = letter , { letter | digit | "_" } ;
% terminal = "'" , character , { character } , "'" 
%          | '"' , character , { character } , '"' ;
 
% lhs = identifier ;
% rhs = identifier
%      | terminal
%      | "[" , rhs , "]"
%      | "{" , rhs , "}"
%      | "(" , rhs , ")"
%      | rhs , "|" , rhs
%      | rhs , "," , rhs ;
 
% rule = lhs , "=" , rhs , ";" ;
% grammar = { rule } ;
% \end{ebnfcode}
% \caption{EBNF определённый через EBNF}
% \label{lst:ebnf}
% \end{listing}

% А вот в листинге \ref{lst:c} на языке C работают русские комменты. Спасибо Pygments и Minted за это.

% \begin{listing}[H]
% \cfile{inc/src/test.c}
% \caption{Пример — test.c} 
% \end{listing}
% \label{lst:c}

% \fi

% % Для вставки реального кода лучше использовать \texttt{\textbackslash lstinputlisting} (который понимает
% % UTF8) и стили \Code{realcode} либо \Code{simplecode} (в зависимости от размера куска).




% Можно также использовать окружение \Code{verbatim}, если \Code{listings} чем-то не
% устраивает. Только следует помнить, что табы в нём <<съедаются>>. Существует так же команда \Code{\textbackslash{}verbatiminput} для вставки файла.

% \begin{verbatim}
% a_b = a + b; // русский комментарий
% if (a_b > 0)
%     a_b = 0;
% \end{verbatim}

%%% Local Variables:
%%% mode: latex
%%% TeX-master: "rpz"
%%% End:
