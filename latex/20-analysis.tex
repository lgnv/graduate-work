

\definecolor{gray}{rgb}{0.4,0.4,0.4}
\definecolor{darkblue}{rgb}{0.0,0.0,0.6}
\definecolor{cyan}{rgb}{0.0,0.6,0.6}
\lstset{
  basicstyle=\ttfamily,
  columns=fullflexible,
  showstringspaces=false,
  commentstyle=\color{gray}\upshape
}

\lstdefinelanguage{XML}
{
  morestring=[b]",
  morestring=[s]{>}{<},
  morecomment=[s]{<?}{?>},
  stringstyle=\color{black},
  identifierstyle=\color{darkblue},
  keywordstyle=\color{cyan},
  morekeywords={xmlns,version,type}% list your attributes here
}

\chapter*{Основная часть}
\addcontentsline{toc}{chapter}{Основная часть}
\renewcommand{\thesection}{\arabic{section}}

\section{Анализ процесса подготовки задач и их формата}
\label{cha:analysis}
%
% % В начале раздела  можно напомнить его цель
%
В данном разделе анализируется процесс подготовки задач, а так же структура и формат задач, публикуемых в системах Codeforces/Timus.
\newline
Процесс подготовки каждой задачи можно разбить на следующие этапы:
\begin{enumerate}
    \item формирование легенды (идеи для развёрнутого условия задачи);
    \item написание условия, подготовка правильных, неверных;
    и неэффективных решений задачи.
    \item написание чекеров и валидаторов, разработка системы тестов.
\end{enumerate}

Для нашей работы наибольший интерес представляют шаги подготовки правильных, неверных и неэффективных решений, а так же написание чекеров, валидатором и системы тестов. 

Изучим эти пункты основываясь на работе "Система   подготовки   олимпиадных   задач   по   про­граммированию  в  УрГУ"(Алексей Самсонов, Александр Ипатов).


\subsection{Авторские решения}
Автор курса должен подготовить по своей задаче набор различных решений: правильных, неверных и неэффективных. Желательно написать хотя бы одно верное решение на языке Java или C\#, поскольку часто решения, написанные на этих языках, работают медленнее их точных аналогов на C++, что следует учитывать при выставлении ограничений по времени.

В задачах, где есть простое, но асимптотически медленное решение, и сложное эффективное решение, которое необходимо придумать и реализовать учащимся данного курса, автор должен определиться, решения с какой асимптотической сложностью должны засчитываться как «верные», а с какой — нет. 


Например, все решения не хуже $O(n^2\log{}n)$ должны засчитываться, а все решения за $O(n^3)$ — нет. Следует реализовать верное решение и асимптотически медленное решение, ускоренное с помощью неасимптотических оптимизаций. Ограничения на размер входных данных следует подбирать таким образом, чтобы неэффективное решение работало дольше эффективного не менее чем в 8-10 раз. При этом ограничение по времени должно превышать время работы верного решения на максимальном тесте в 2-3 раза.

Отдельное внимание следует уделить формально неверным решениям, которые, однако, могут находить правильный ответ для большинства входных данных. Во многих задачах хороший результат дают случайные алгоритмы. Команда разработчиков должна реализовать такие решения и проверить, что они не проходят некоторые из подготовленных тестов.

\subsection{Чекеры}
Чекеры необходимы в тех ситуациях, когда ответ на задачу для некоторых входных данных неоднозначен. В частности, чекеры активно используются в задачах, где помимо ответа нужно выдать сертификат — информацию, по которой легко убедиться в корректности такого ответа. 

Например, в задаче на поиск минимума функции ответом служит искомый минимум, а сертификатом — аргумент функции, при котором этот минимум достигается; в задаче на поиск кратчайшего расстояния между двумя вершинами графа ответом служит длина кратчайшего пути между вершинами, а сертификатом — сам этот путь. 

Иногда требование выдачи сертификата усложняет решение задачи. Однако это требование позволяет подготовить задачу надёжнее и сделать процесс подготовки более удобным: с помощью чекера можно удостовериться в том, что авторское решение находит корректный ответ; в противном случае, чекер выдаёт информацию о том, где именно в ответе содержится ошибка, что позволяет быстрее обнаружить ошибку в эталонном решении. 

Обратите внимание, что хотя чекер не решает задачу самостоятельно, за счёт сертификата он может проверить, какой из двух ответов лучше. Также стоит отметить, что если авторское решение выдаёт некорректный ответ на какой-либо тест, то при проверке этого решения чекер выдаст вердикт «неправильный ответ», даже если ответ, выданный авторским решением, будет в точности совпадать с ответом в наборе тестов!

\subsection{Тесты}

Процесс подготовки тестов к любой задаче начинается с создания валидатора — программы, проверяющей корректность входных данных. Валидатор проверяет, что формат файла со входными данными в точности соответствует описанному в условии: везде в файле стоит нужное количество пробелов и переводов строк, в конце файла нет лишней информации, все данные удовлетворяют указанным ограничениям. В валидаторах иногда проверяются и более сложные ограничения — например, связность заданного в файле графа или отсутствие общих точек у двух геометрических фигур.

Автор задачи старается разработать максимально полный набор тестов. Он должен включать множество небольших тестов на разные случаи, ответы на которые легко проверить вручную, тесты на крайние случаи, большие тесты, сгенерированные по шаблону, а также достаточное количество случайных тестов разного размера. Также следует убедиться в том, что асимптотически медленные, эвристические и случайные решения не проходят какие-либо из подготовленных тестов. Для этого нужно либо специально готовить тесты против таких решений, либо воспользоваться техникой стресс-тестирования.

Для проведения стресс-тестирования пишется генератор случайных тестов и чекер. Далее генерируется случайный тест, на нём запускаются два решения (обычно — эталонное и неверное), полученные ответы сравниваются. Если ответы расходятся, то тест против неверного решения готов. Если же ответы совпадают, генерируется новый случайный тест, и так до тех пор, пока не обнаружится расхождение. Стресс-тестирование позволяет находить неочевидные тесты против эвристик, увеличивает надёжность подготовки задачи: если в задаче требуется выдать сертификат, можно запускать стресс-тестирование эталонного решения и проверять, что оно пройдёт все сгенерированные тесты. Интересной техникой подготовки тестов является стресс-тестирование эталонных решений с малыми изменениями. Это позволяет быстро строить сложные тесты на крайние случаи и лучше исследовать задачу.

\subsection{Разработка задачи}
Для разработки задачи используется система \\polygon.codeforces.com

Polygon поддерживает весь цикл разработки:
\begin{enumerate}
    \item написание постановки задачи;
    \item подготовка тестовых данных (поддерживаются генераторы);
    \item модельные решения (в том числе правильные и заведомо неправильные);
    \item автоматическая проверка.
\end{enumerate}

Результатом разработки задачи в данном сервисе является архив, содержимое которого представлено в Листинге 1.1
\begin{lstlisting}[caption={Содержимое архива задачи}]
files
    /...
scripts
    /...
solutions
    /...
statements
    /.html
        /russian
            /problem.html
            /problem-statements.css
    /russian
        /problem.tex
        /problem-properties.json
statement-sections
    /...
tests
    /01
    /01.a
    /02
    /02.a
    /...
check.cpp
check.exe
problem.xml
\end{lstlisting}

Наибольший интерес для нашей работы представляют следующие пункты: 
\begin{itemize}
\item директория solutions, в которой находятся авторские решения;
\item директория statements, в которой находятся постановка задачи: html-разметка, pdf-файл, latex-разметка на которых находится легенда задачи, условия, примеры входных и выходных данных и тд;
\item директория tests, в которой находятся входные данные тестов и их ответы;
\item файл check.cpp;
\item файл problem.xml.
\end{itemize}

Разберем некоторые из этих пункты подробнее.

\textbf{Файл problem.xml} представляет интерес благодаря тому, что в нем мы можем найти информацию, которая соотносит авторские решения и их статус: accepted, main, wrong-answer и т.д. 

Помимо статуса решения есть так же информация об языке, на котором оно написано и о версии используемого компилятора.

Пример данного файла представлен в листинге 1.2.

\begin{lstlisting}[caption={Пример информации об авторских решениях}, language={XML}]
<solutions>
    <solution tag="accepted">
        <source path="solutions/AGProgressionCpp.cpptype="cpp.g++14"/>
        <binary path="solutions/AGProgressionCpp.exetype="exe.win32"/>
    </solution>
    <solution tag="wrong-answer">
        <source path="solutions/Formula.cs" type="csharp.mono"/>
        <binary path="solutions/Formula.exe" type="exe.win32"/>
    </solution>
    <solution tag="main">
        <source path="solutions/Main.cs" type="csharp.mono"/>
        <binary path="solutions/Main.exe" type="exe.win32"/>
    </solution>
</solutions>
\end{lstlisting}

\textbf{Файл check.cpp} является тем самым чекером для текущей задачи, описанный в пункте 1.1.2

\textbf{Директория statements} как уже было описано выше содержит визуальное представление данной задачи. Директорий представлена в листинге 1.3.
\\ \\
\begin{lstlisting}[caption={Пример содержимого директории statements}]
.html
    /russian                     
        /problem.html
        /problem-statements.css
russian
    /problem.tex
    /problem-properties.json
\end{lstlisting}


\subsection{Результат анализа}
По итогам проведенного исследования процесса подготовки задач и их формата, были изучены шаги и главные тонкости составления задач, а так же выделены проблемы и важные места, которые необходимо будет учесть в реализации собственной системы:
\begin{enumerate}
    \item вся ответственность за написание тестов и различных авторских решений остается на создателе курса. Система Ulearn будет предоставлять возможность размещения задач и проверки решений;
    \item Опираясь на то, что в каталоге задачи всегда присутствует информация о визуальном представлении для пользователя, то система должна автоматически генерировать его без дополнительных действий со стороны автора курса;
\end{enumerate}



% Обратите внимание, что включается не ../dia/..., а inc/dia/...
% В Makefile есть соответствующее правило для inc/dia/*.pdf, которое
% берет исходные файлы из ../dia в этом случае.

% \begin{figure}
%   \centering
%   \includegraphics[width=\textwidth]{figures/pic01}
%   \caption{Рисунок}
%   \label{fig:fig01}
% \end{figure}

% В \cite{Pup09} указано, что...

% Кстати, про картинки. Во-первых, для фигур следует использовать \texttt{[ht]}. Если и после этого картинки вставляются <<не по ГОСТ>>, т.е. слишком далеко от места ссылки,~--- значит у вас в РПЗ \textbf{слишком мало текста}! Хотя и ужасный параметр \texttt{!ht} у окружения \texttt{figure} тоже никто не отменял, только при его использовании документ получается страшный, как в ворде, поэтому просьба так не делать по возможности.


% Известны следующие подходы...

% \begin{enumerate}
% \item Перечисление с номерами.
% \item Номера первого уровня. Да, ГОСТ требует именно так~--- сначала буквы, на втором уровне~--- цифры.
% Чуть ниже будет вариант <<нормальной>> нумерации и советы по её изменению.
% Да, мне так нравится: на первом уровне выравнивание элементов как у обычных абзацев. Проверим теперь вложенные списки.
% \begin{enumerate}
% \item Номера второго уровня.
% \item Номера второго уровня. Проверяем на длииииной-предлиииииииииинной строке, что получается.... Сойдёт.
% \end{enumerate}
% \item По мнению Лукьяненко, человеческий мозг старается подвести любую проблему к выбору
%   из трех вариантов.
% \item Четвёртый (и последний) элемент списка.
% \end{enumerate}

% Теперь мы покажем, как изменить нумерацию на «нормальную», если вам этого захочется. Пара команд в начале документа поможет нам.

% \renewcommand{\labelenumi}{\arabic{enumi})}
% \renewcommand{\labelenumii}{\asbuk{enumii})}

% \begin{enumerate}
% \item Изменим нумерацию на более привычную...
% \item ... нарушим этим гост.
% \begin{enumerate}
% \item Но, пожалуй, так лучше.
% \end{enumerate}
% \end{enumerate}

% В заключение покажем произвольные маркеры в списках. Для них нужен пакет \textbf{enumerate}.
% \begin{enumerate}[1.]
% \item Маркер с арабской цифрой и с точкой.
% \item Маркер с арабской цифрой и с точкой.
% \begin{enumerate}[I.]
% \item Римская цифра с точкой.
% \item Римская цифра с точкой.
% \end{enumerate}
% \end{enumerate}

% В отчётах могут быть и таблицы~--- см. табл.~\ref{tab:tabular} и~\ref{tab:longtable}.
% Небольшая таблица делается при помощи \Code{tabular} внутри \Code{table} (последний
% полностью аналогичен \Code{figure}, но добавляет другую подпись).

% \begin{table}[ht]
%   \caption{Пример короткой таблицы с длинным названием на много длинных-длинных строк}
%   \begin{tabular}{|r|c|c|c|l|}
%   \hline
%   Тело      & $F$ & $V$  & $E$ & $F+V-E-2$ \\
%   \hline
%   Тетраэдр  & 4   & 4    & 6   & 0         \\
%   Куб       & 6   & 8    & 12  & 0         \\
%   Октаэдр   & 8   & 6    & 12  & 0         \\
%   Додекаэдр & 20  & 12   & 30  & 0         \\
%   Икосаэдр  & 12  & 20   & 30  & 0         \\
%   \hline
%   Эйлер     & 666 & 9000 & 42  & $+\infty$ \\
%   \hline
%   \end{tabular}
%   \label{tab:tabular}
% \end{table}

% Для больших таблиц следует использовать пакет \Code{longtable}, позволяющий создавать
% таблицы на несколько страниц по ГОСТ.

% Для того, чтобы длинный текст разбивался на много строк в пределах одной ячейки, надо в
% качестве ее формата задавать \texttt{p} и указывать явно ширину: в мм/дюймах
% (\texttt{110mm}), относительно ширины страницы (\texttt{0.22\textbackslash textwidth})
% и~т.п.

% Можно также использовать уменьшенный шрифт~--- но, пожалуйста, тогда уж во \textbf{всей}
% таблице сразу.

% \begin{center}
%   \begin{longtable}{|p{0.40\textwidth}|c|p{0.30\textwidth}|}
%     \caption{Пример длинной таблицы с длинным названием на много длинных-длинных строк}
%     \label{tab:longtable}
%     \\ \hline
%     Вид шума & Громкость, дБ & Комментарий \\
%     \hline \endfirsthead
%     \subcaption{Продолжение таблицы~\ref{tab:longtable}}
%     \\ \hline \endhead
%     \hline \subcaption{Продолжение на след. стр.}
%     \endfoot
%     \hline \endlastfoot
%     Порог слышимости             & 0     &                                                \\
%     \hline
%     Шепот в тихой библиотеке     & 30    &                                                \\
%     Обычный разговор             & 60-70 &                                                \\
%     Звонок телефона              & 80    & \small{Конечно, это было до эпохи мобильников} \\
%     Уличный шум                  & 85    & \small{(внутри машины)}                        \\
%     Гудок поезда                 & 90    &                                                \\
%     Шум электрички               & 95    &                                                \\
%     \hline
%     Порог здоровой нормы         & 90-95 & \small{Длительное пребывание на более
%     громком шуме может привести к ухудшению слуха}                                        \\
%     \hline
%     Мотоцикл                     & 100   &                                                \\
%     Power Mower                  & 107   & \small{(модель бензокосилки)}                  \\
%     Бензопила                    & 110   & \small{(Doom в целом вреден для здоровья)}     \\
%     Рок-концерт                  & 115   &                                                \\
%     \hline
%     Порог боли                   & 125   & \small{feel the pain}                          \\
%     \hline
%     Клепальный молоток           & 125   & \small{(автор сам не знает, что это)}          \\
%     \hline
%     Порог опасности              & 140   & \small{Даже кратковременное пребывание на
%     шуме большего уровня может привести к необратимым последствиям}                       \\
%     \hline
%     Реактивный двигатель         & 140   &                                                \\
%                                  & 180   & \small{Необратимое полное повреждение
%                                  слуховых органов}                                        \\
%     Самый громкий возможный звук & 194   & \small{Интересно, почему?..}                   \\
%   \end{longtable}
% \end{center}

%%% Local Variables:
%%% mode: latex
%%% TeX-master: "rpz"
%%% End:
