\Introduction
Актуальность выбранной темы дипломной работы обусловлена тем, что авторам курсов был необходим инструмент для формирования и публикации своего набора задач для студентов университетов, школьников и всех желающих изучить материал курса с выбором языка из множества допустимых. Преимущество системы Ulearn перед такими системами как Codeforces и Timus заключается в возможности:
\begin{itemize}
\item переиспользовать готовые задачи, подготовленные ранее в polygon, а не переводить в формат Ulearn;
\item не изучать автору формат задач Ulearn, если он знаком с форматом Codeforces;
\item проводить проверку кода студента на понятность, стиль принятый для текущего языка;
\item регулировать стоимость каждого задания в баллах;
\item размещать теорию и практику курса в одном месте.
\end{itemize}

Целью дипломной работы является реализация поддержки задач в формате Codeforces в образовательной платформе Ulearn: возможность размещения таких курсов, автопроверки решений по задачам.

Для достижения поставленной цели необходимо решить следующие задачи:

\begin{itemize}
\item проанализировать процесс подготовки задач, а так же структуру и формат задач, публикуемых в системах Codeforces/Timus;
\item изучить текущую инфраструктуру и способ формирования курсов в образовательной системе Ulearn;
\item спроектировать решение в рамках принятых в системе Ulearn паттернов;
\item реализовать это решение;
\item проверить работоспособность и удобство использования для авторов курсов.
\end{itemize}

