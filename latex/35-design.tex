\definecolor{gray}{rgb}{0.4,0.4,0.4}
\definecolor{darkblue}{rgb}{0.0,0.0,0.6}
\definecolor{cyan}{rgb}{0.0,0.6,0.6}
\lstloadlanguages{C,C++,csh,Java}

\definecolor{red}{rgb}{0.6,0,0} 
\definecolor{blue}{rgb}{0,0,0.6}
\definecolor{green}{rgb}{0,0.8,0}
\definecolor{cyan}{rgb}{0.0,0.6,0.6}

\lstset{
  basicstyle=\ttfamily,
  columns=fullflexible,
  showstringspaces=false,
  commentstyle=\color{gray}\upshape
}

\lstdefinelanguage{XML}
{
  morestring=[b]",
  morestring=[s]{>}{<},
  morecomment=[s]{<?}{?>},
  stringstyle=\color{black},
  identifierstyle=\color{darkblue},
  keywordstyle=\color{cyan},
  morekeywords={xmlns,version,type}% list your attributes here
}



\section{Проектирование решения}
\label{cha:design}


В этом разделе спроектировано решение в рамках принятых в системе Ulearn паттернов.
\\


Исходя из рассмотренной структуры в пункте 1.2.1 становится ясно, что задачи должны находиться на слайде типа Exercise. Дальнейшие шаги имеют несколько вариантов реализации. Рассмотрим их и выделим плюсы и минусы каждого.

\begin{enumerate}
\item Создать новый тип задач, который будет независим от уже существующих в системе. Прописать его типизацию в xsd-схеме. Создать сущность для обработки конфигурации, реализующую контракт ISlide. А так же реализовать систему проверки;
\\ \textbf{Плюсы:}
     \begin{enumerate}                  
         \item Автору при заполнении xml-файла не будет подсказываться ничего лишнего, относящегося к другим типам задач.
    \end{enumerate}
\\ \textbf{Минусы:}
    \begin{enumerate}
        \item Нарушение принципа DRY, так как большую часть логики можно было переиспользовать из уже реализованной в системе.
    \end{enumerate}
\item Внести логику реализуемых задач внутрь сущностей типа UniversalExercise;
\\ \textbf{Плюсы:}
     \begin{enumerate}                  
         \item Логика системы переиспользуется;
         \item сокращается количество кода, которое необходимо написать.
    \end{enumerate}
\\ \textbf{Минусы:}
    \begin{enumerate}
        \item Нарушение принципа SRP, так как в нашей конфигурации будет поля, которые не нужны в UniversalExercise и наоборот.
    \end{enumerate}
    
\item Создать новый тип задач, сделать его наследником типа UniversalExercise и в xsd-схеме и в обработке конфигурации (так как остальные специализированы только на проверке задач, реализованных на языке C\#);
\\ \textbf{Плюсы:}
     \begin{enumerate}                  
         \item Логика системы переиспользуется;
         \item сокращается количество кода, которое необходимо написать.
    \end{enumerate}
\\ \textbf{Минусы:}
    \begin{enumerate}
        \item Автор при создании слайда в xml будет видеть в подсказки, которые не участвуют в конфигурации нашей задачи.
    \end{enumerate}
\end{enumerate}

Проанализировав данные варианты было принято решение совместить вариант а) и вариант в). В результате получается решение: \\
Создать новый тип задач, сделать его наследником типа UniversalExercise в обработке конфигурации, но прописать для него независимую типизацию в xsd-схеме. Таким образом единственный минус варианта в) будет решен.			
			
% \paragraph{Проверка} параграфа. Вроде работает.
% \paragraph{Вторая проверка} параграфа. Опять работает.

% Вот.

% \begin{itemize}
% \item Это список с <<палочками>>.
% \item Хотя он и не по ГОСТ, кажется.
% \end{itemize}

% \begin{enumerate}
% \item Поэтому для списка, начинающегося с заглавной буквы, лучше список с цифрами.
% \end{enumerate}

% Формула \ref{F:F1} совершено бессмысленна.

% %Кстати, при каких-то условиях <<удавалось>> получить двойный скобки вокруг номеров формул. Вопрос исследуется.

% \begin{equation}
% a= cb
% \label{F:F1}
% \end{equation}


% Окружение \texttt{cases} опять работает (см. \ref{F:F2}), спасибо И. Короткову за исправления..


% \begin{equation}
% a= \begin{cases}
%  3x + 5y + z, \mbox{если хорошо} \\
%  7x - 2y + 4z, \mbox{если плохо}\\
%  -6x + 3y + 2z, \mbox{если совсем плохо}
% \end{cases}
% \label{F:F2}
% \end{equation}

% \section{Подсистема всякой ерунды}

% Культурная вставка dot-файлов через утилиту dot2tex (рис.~\ref{fig:fig02}).

% \begin{figure}
%   \centering
% % [width=0.5\textwidth] --- регулировка ширины картинки
%   \includegraphics{figures/pic01}
%   \caption{Рисунок}
%   \label{fig:fig02}
% \end{figure}


% \subsection{Блок-схема всякой ерунды}

% \subsubsection*{Кстати о заголовках}

% У нас есть и \Code{subsubsection}. Только лучше её не нумеровать.

%%% Local Variables:
%%% mode: latex
%%% TeX-master: "rpz"
%%% End:
