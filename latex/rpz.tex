%% Преамбула TeX-файла

% 1. Стиль и язык
\documentclass[utf8x]{G7-32} % Стиль (по умолчанию будет 14pt)
\usepackage[T2A]{fontenc}
\usepackage[russian]{babel}
\usepackage{amssymb}
% Остальные стандартные настройки убраны в preamble.inc.tex.
\include{preamble.inc}

% Настройки листингов.
\include{listings.inc}

% Полезные макросы листингов.
\include{macros.inc}

\newtheorem{definition}{Определение}
\newtheorem{theorem}{Теорема}

\begin{document}
% \begin{tabular}{ll}
% amssymb:        & $\mathbb{R}$
% \end{tabular}

\frontmatter % выключает нумерацию ВСЕГО; здесь начинаются ненумерованные главы: реферат, введение, глоссарий, сокращения и прочее.

% Команды \breakingbeforechapters и \nonbreakingbeforechapters
% управляют разрывом страницы перед главами.
% По-умолчанию страница разрывается.

% \nobreakingbeforechapters
% \breakingbeforechapters

 % \documentclass[a4paper, 14pt]{extarticle}


% %% Language and font encodings
% \usepackage[english, russian]{babel}
% \usepackage[utf8]{inputenc}
% \usepackage[T1]{fontenc}
% \usepackage{indentfirst}
% \usepackage{setspace}
% \usepackage{float}

% %% Sets page size and margins
% \usepackage[a4paper,top=2cm,bottom=2cm,left=3cm,right=1.5cm,margin=15mm, lmargin=30mm]{geometry}

% %% Useful packages
% \usepackage{amsmath,amssymb}
% \usepackage{graphicx}
% \usepackage[colorinlistoftodos]{todonotes}
% \usepackage[colorlinks=true, allcolors=blue]{hyperref}
% \usepackage{amsmath}
% \DeclareMathOperator*{\argmax}{arg\,max}
% \DeclareMathOperator*{\argmin}{arg\,min}
% \setstretch{1.5}


\begin{titlepage}
\clearpage\maketitle
\thispagestyle{empty}

\begin{centering}
    {
    МИНИСТЕРСТВО НАУКИ И ВЫСШЕГО ОБРАЗОВАНИЯ РОССИЙСКОЙ ФЕДЕРАЦИИ\\
    Федеральное государственное автономное образовательное учреждение
    высшего образования\\
    УРАЛЬСКИЙ ФЕДЕРАЛЬНЫЙ УНИВЕРСИТЕТ\\
    имени первого Президента России Б.Н. Ельцина\\ \vspace{0.5cm}
    ИНСТИТУТ ЕСТЕСТВЕННЫХ НАУК И МАТЕМАТИКИ \\}
    
\vfill
{
  Департамент математики, механики и компьютерных наук\\
\bigskip\bigskip
  \textbf{Реализация поддержки задач в формате codeforces в образовательной платформе Ulearn
  \\}
\bigskip
  {Направление подготовки 02.03.02 «Фундаментальная информатика и информационные технологии»}
\\ 	\vspace{0.5cm}
}
\vfill

\begin{minipage}[t]{.50\textwidth}
Директор департамента:\\
к. ф.-м.н., доц Е.С. Пьянзина \bigskip

\underline{\hspace{5.5cm}}\\

Нормоконтролер: \\
О.А. Суслова\\
\bigskip
\underline{\hspace{5.5cm}}\\
\end{minipage}%
%
\begin{minipage}[t]{.50\textwidth}
\begin{centering}
Выпускная квалификационная
работа бакалавра\\
\textbf{Логинова Александра Владиславовича}\\
\bigskip
\underline{\hspace{5.5cm}}\\
\bigskip\bigskip

Научный руководитель: \\
ст.преп., П. В. Егоров
\underline{\hspace{5.5cm}}\\
\bigskip
асс., А. Н. Федоров\\
\underline{\hspace{5.5cm}}\\
\bigskip
Научный соруководитель: \\
канд. физ.-мат. наук, доц.,\\ О. В. Расин\\
\underline{\hspace{5.5cm}}\\
\bigskip
\end{centering}
\end{minipage}

\vfill

{Екатеринбург\\2021}

\end{centering}
\end{titlepage}



 % Также можно использовать \Referat, как в оригинале
\begin{abstract}
Выпускная квалификационная работа состоит из введения, 4 разделов и 16 подразделов, заключения, списка используемых источников состоящего из 7 источников. Работа изложена на 36 листах печатного текста, содержит 22 листинга.

Ключевые слова: POLYGON, TIMUS, CODEFORCES, ОБРАЗОВАТЕЛЬНЫЕ КУРСЫ, ULEARN, DOCKER, ОЛИМПИАДНЫЕ ЗАДАЧИ, АВТОПРОВЕРКА.

Целью дипломной работы является реализация поддержки задач в формате Codeforces в образовательной платформе Ulearn: возможность размещения таких курсов, автопроверки решений по задачам.

Методы исследования: анализ, синтез, наблюдение, сравнение. Источник данных - документация формата задач, статьи авторов, платформы выполняющие подобную задачу.

В ходе работы было разработано и внедрено в сервис Ulearn в компании Скб Контур решение позволяющее поддерживать задачи в формате Codeforces.

\end{abstract}

%%% Local Variables: 
%%% mode: latex
%%% TeX-master: "rpz"
%%% End: 


\tableofcontents

 \Defines % Необходимые определения. Вряд ли понадобться
В настоящей работе применяются следующие термины
\begin{description}
\item[Ulearn] платформа с интерактивными онлайн-курсами по программированию
\item[Codeforces] проект, объединяющий людей, которые интересуются и участвуют в соревнованиях по программированию. Является как социальной сетью, посвященной программированию и соревнованиям так и площадкой, где регулярно проводятся соревнования
\item[Timus Online Judge] крупнейший в России архив задач по программированию с автоматической проверяющей системой.
\item[Формат задач Codeforces] принятый стандарт внутреннего описания олимпиадных задач. Распространяется также на Timus Online Judge
\item[Polygon] платформа для создания задач по программированию
\item[Docker] программное обеспечение с открытым исходным кодом, применяемое для разработки, тестирования, доставки и запуска веб-приложений в средах с поддержкой контейнеризации
\item[Чеккер] программа, выполняющая задачу проверки корректности ответа

\item[Don’t repeat yourself] принцип разработки программного обеспечения, нацеленный на снижение повторения информации различного рода, особенно в системах со множеством слоёв абстрагирования.
\item[Single-responsibility principle] принцип ООП, обозначающий, что каждый объект должен иметь одну ответственность и эта ответственность должна быть полностью инкапсулирована в класс. Все его поведения должны быть направлены исключительно на обеспечение этой ответственности.
\end{description}
%%% Local Variables:
%%% mode: latex
%%% TeX-master: "rpz"
%%% End:

 
\Abbreviations %% Список обозначений и сокращений в тексте
В настоящей работе применяются следующие обозначения и сокращения 
\begin{description}
\item[DRY] Don’t repeat yourself
\item[SRP] Single-responsibility principle
   
\end{description}
%%% Local Variables:
%%% mode: latex
%%% TeX-master: "rpz"
%%% End:


\Introduction
Актуальность выбранной темы дипломной работы обусловлена тем, что авторам курсов был необходим инструмент для формирования и публикации своего набора задач для студентов университетов, школьников и всех желающих изучить материал курса с выбором языка из множества допустимых. Преимущество системы Ulearn перед такими системами как Codeforces и Timus заключается в возможности:
\begin{itemize}
\item переиспользовать готовые задачи, подготовленные ранее в polygon, а не переводить в формат Ulearn;
\item не изучать автору формат задач Ulearn, если он знаком с форматом Codeforces;
\item проводить проверку кода студента на понятность, стиль принятый для текущего языка;
\item регулировать стоимость каждого задания в баллах;
\item размещать теорию и практику курса в одном месте.
\end{itemize}

Целью дипломной работы является реализация поддержки задач в формате Codeforces в образовательной платформе Ulearn: возможность размещения таких курсов, автопроверки решений по задачам.

Для достижения поставленной цели необходимо решить следующие задачи:

\begin{itemize}
\item проанализировать процесс подготовки задач, а так же структуру и формат задач, публикуемых в системах Codeforces/Timus;
\item изучить текущую инфраструктуру и способ формирования курсов в образовательной системе Ulearn;
\item спроектировать решение в рамках принятых в системе Ulearn паттернов;
\item реализовать это решение;
\item проверить работоспособность и удобство использования для авторов курсов.
\end{itemize}



% \include{intro} было для преддипломной практики

\mainmatter % это включает нумерацию глав и секций в документе ниже


\definecolor{gray}{rgb}{0.4,0.4,0.4}
\definecolor{darkblue}{rgb}{0.0,0.0,0.6}
\definecolor{cyan}{rgb}{0.0,0.6,0.6}
\lstset{
  basicstyle=\ttfamily,
  columns=fullflexible,
  showstringspaces=false,
  commentstyle=\color{gray}\upshape
}

\lstdefinelanguage{XML}
{
  morestring=[b]",
  morestring=[s]{>}{<},
  morecomment=[s]{<?}{?>},
  stringstyle=\color{black},
  identifierstyle=\color{darkblue},
  keywordstyle=\color{cyan},
  morekeywords={xmlns,version,type}% list your attributes here
}

\chapter*{Основная часть}
\addcontentsline{toc}{chapter}{Основная часть}
\renewcommand{\thesection}{\arabic{section}}

\section{Анализ процесса подготовки задач и их формата}
\label{cha:analysis}
%
% % В начале раздела  можно напомнить его цель
%
В данном разделе анализируется процесс подготовки задач, а так же структура и формат задач, публикуемых в системах Codeforces/Timus.
\newline
Процесс подготовки каждой задачи можно разбить на следующие этапы:
\begin{enumerate}
    \item формирование легенды (идеи для развёрнутого условия задачи);
    \item написание условия, подготовка правильных, неверных;
    и неэффективных решений задачи.
    \item написание чекеров и валидаторов, разработка системы тестов.
\end{enumerate}

Для нашей работы наибольший интерес представляют шаги подготовки правильных, неверных и неэффективных решений, а так же написание чекеров, валидатором и системы тестов. 

Изучим эти пункты основываясь на работе "Система   подготовки   олимпиадных   задач   по   про­граммированию  в  УрГУ"(Алексей Самсонов, Александр Ипатов).


\subsection{Авторские решения}
Автор курса должен подготовить по своей задаче набор различных решений: правильных, неверных и неэффективных. Желательно написать хотя бы одно верное решение на языке Java или C\#, поскольку часто решения, написанные на этих языках, работают медленнее их точных аналогов на C++, что следует учитывать при выставлении ограничений по времени.

В задачах, где есть простое, но асимптотически медленное решение, и сложное эффективное решение, которое необходимо придумать и реализовать учащимся данного курса, автор должен определиться, решения с какой асимптотической сложностью должны засчитываться как «верные», а с какой — нет. 


Например, все решения не хуже $O(n^2\log{}n)$ должны засчитываться, а все решения за $O(n^3)$ — нет. Следует реализовать верное решение и асимптотически медленное решение, ускоренное с помощью неасимптотических оптимизаций. Ограничения на размер входных данных следует подбирать таким образом, чтобы неэффективное решение работало дольше эффективного не менее чем в 8-10 раз. При этом ограничение по времени должно превышать время работы верного решения на максимальном тесте в 2-3 раза.

Отдельное внимание следует уделить формально неверным решениям, которые, однако, могут находить правильный ответ для большинства входных данных. Во многих задачах хороший результат дают случайные алгоритмы. Команда разработчиков должна реализовать такие решения и проверить, что они не проходят некоторые из подготовленных тестов.

\subsection{Чекеры}
Чекеры необходимы в тех ситуациях, когда ответ на задачу для некоторых входных данных неоднозначен. В частности, чекеры активно используются в задачах, где помимо ответа нужно выдать сертификат — информацию, по которой легко убедиться в корректности такого ответа. 

Например, в задаче на поиск минимума функции ответом служит искомый минимум, а сертификатом — аргумент функции, при котором этот минимум достигается; в задаче на поиск кратчайшего расстояния между двумя вершинами графа ответом служит длина кратчайшего пути между вершинами, а сертификатом — сам этот путь. 

Иногда требование выдачи сертификата усложняет решение задачи. Однако это требование позволяет подготовить задачу надёжнее и сделать процесс подготовки более удобным: с помощью чекера можно удостовериться в том, что авторское решение находит корректный ответ; в противном случае, чекер выдаёт информацию о том, где именно в ответе содержится ошибка, что позволяет быстрее обнаружить ошибку в эталонном решении. 

Обратите внимание, что хотя чекер не решает задачу самостоятельно, за счёт сертификата он может проверить, какой из двух ответов лучше. Также стоит отметить, что если авторское решение выдаёт некорректный ответ на какой-либо тест, то при проверке этого решения чекер выдаст вердикт «неправильный ответ», даже если ответ, выданный авторским решением, будет в точности совпадать с ответом в наборе тестов!

\subsection{Тесты}

Процесс подготовки тестов к любой задаче начинается с создания валидатора — программы, проверяющей корректность входных данных. Валидатор проверяет, что формат файла со входными данными в точности соответствует описанному в условии: везде в файле стоит нужное количество пробелов и переводов строк, в конце файла нет лишней информации, все данные удовлетворяют указанным ограничениям. В валидаторах иногда проверяются и более сложные ограничения — например, связность заданного в файле графа или отсутствие общих точек у двух геометрических фигур.

Автор задачи старается разработать максимально полный набор тестов. Он должен включать множество небольших тестов на разные случаи, ответы на которые легко проверить вручную, тесты на крайние случаи, большие тесты, сгенерированные по шаблону, а также достаточное количество случайных тестов разного размера. Также следует убедиться в том, что асимптотически медленные, эвристические и случайные решения не проходят какие-либо из подготовленных тестов. Для этого нужно либо специально готовить тесты против таких решений, либо воспользоваться техникой стресс-тестирования.

Для проведения стресс-тестирования пишется генератор случайных тестов и чекер. Далее генерируется случайный тест, на нём запускаются два решения (обычно — эталонное и неверное), полученные ответы сравниваются. Если ответы расходятся, то тест против неверного решения готов. Если же ответы совпадают, генерируется новый случайный тест, и так до тех пор, пока не обнаружится расхождение. Стресс-тестирование позволяет находить неочевидные тесты против эвристик, увеличивает надёжность подготовки задачи: если в задаче требуется выдать сертификат, можно запускать стресс-тестирование эталонного решения и проверять, что оно пройдёт все сгенерированные тесты. Интересной техникой подготовки тестов является стресс-тестирование эталонных решений с малыми изменениями. Это позволяет быстро строить сложные тесты на крайние случаи и лучше исследовать задачу.

\subsection{Разработка задачи}
Для разработки задачи используется система \\polygon.codeforces.com

Polygon поддерживает весь цикл разработки:
\begin{enumerate}
    \item написание постановки задачи;
    \item подготовка тестовых данных (поддерживаются генераторы);
    \item модельные решения (в том числе правильные и заведомо неправильные);
    \item автоматическая проверка.
\end{enumerate}

Результатом разработки задачи в данном сервисе является архив, содержимое которого представлено в Листинге 1.1
\begin{lstlisting}[caption={Содержимое архива задачи}]
files
    /...
scripts
    /...
solutions
    /...
statements
    /.html
        /russian
            /problem.html
            /problem-statements.css
    /russian
        /problem.tex
        /problem-properties.json
statement-sections
    /...
tests
    /01
    /01.a
    /02
    /02.a
    /...
check.cpp
check.exe
problem.xml
\end{lstlisting}

Наибольший интерес для нашей работы представляют следующие пункты: 
\begin{itemize}
\item директория solutions, в которой находятся авторские решения;
\item директория statements, в которой находятся постановка задачи: html-разметка, pdf-файл, latex-разметка на которых находится легенда задачи, условия, примеры входных и выходных данных и тд;
\item директория tests, в которой находятся входные данные тестов и их ответы;
\item файл check.cpp;
\item файл problem.xml.
\end{itemize}

Разберем некоторые из этих пункты подробнее.

\textbf{Файл problem.xml} представляет интерес благодаря тому, что в нем мы можем найти информацию, которая соотносит авторские решения и их статус: accepted, main, wrong-answer и т.д. 

Помимо статуса решения есть так же информация об языке, на котором оно написано и о версии используемого компилятора.

Пример данного файла представлен в листинге 1.2.

\begin{lstlisting}[caption={Пример информации об авторских решениях}, language={XML}]
<solutions>
    <solution tag="accepted">
        <source path="solutions/AGProgressionCpp.cpptype="cpp.g++14"/>
        <binary path="solutions/AGProgressionCpp.exetype="exe.win32"/>
    </solution>
    <solution tag="wrong-answer">
        <source path="solutions/Formula.cs" type="csharp.mono"/>
        <binary path="solutions/Formula.exe" type="exe.win32"/>
    </solution>
    <solution tag="main">
        <source path="solutions/Main.cs" type="csharp.mono"/>
        <binary path="solutions/Main.exe" type="exe.win32"/>
    </solution>
</solutions>
\end{lstlisting}

\textbf{Файл check.cpp} является тем самым чекером для текущей задачи, описанный в пункте 1.1.2

\textbf{Директория statements} как уже было описано выше содержит визуальное представление данной задачи. Директорий представлена в листинге 1.3.
\\ \\
\begin{lstlisting}[caption={Пример содержимого директории statements}]
.html
    /russian                     
        /problem.html
        /problem-statements.css
russian
    /problem.tex
    /problem-properties.json
\end{lstlisting}


\subsection{Результат анализа}
По итогам проведенного исследования процесса подготовки задач и их формата, были изучены шаги и главные тонкости составления задач, а так же выделены проблемы и важные места, которые необходимо будет учесть в реализации собственной системы:
\begin{enumerate}
    \item вся ответственность за написание тестов и различных авторских решений остается на создателе курса. Система Ulearn будет предоставлять возможность размещения задач и проверки решений;
    \item Опираясь на то, что в каталоге задачи всегда присутствует информация о визуальном представлении для пользователя, то система должна автоматически генерировать его без дополнительных действий со стороны автора курса;
\end{enumerate}



% Обратите внимание, что включается не ../dia/..., а inc/dia/...
% В Makefile есть соответствующее правило для inc/dia/*.pdf, которое
% берет исходные файлы из ../dia в этом случае.

% \begin{figure}
%   \centering
%   \includegraphics[width=\textwidth]{figures/pic01}
%   \caption{Рисунок}
%   \label{fig:fig01}
% \end{figure}

% В \cite{Pup09} указано, что...

% Кстати, про картинки. Во-первых, для фигур следует использовать \texttt{[ht]}. Если и после этого картинки вставляются <<не по ГОСТ>>, т.е. слишком далеко от места ссылки,~--- значит у вас в РПЗ \textbf{слишком мало текста}! Хотя и ужасный параметр \texttt{!ht} у окружения \texttt{figure} тоже никто не отменял, только при его использовании документ получается страшный, как в ворде, поэтому просьба так не делать по возможности.


% Известны следующие подходы...

% \begin{enumerate}
% \item Перечисление с номерами.
% \item Номера первого уровня. Да, ГОСТ требует именно так~--- сначала буквы, на втором уровне~--- цифры.
% Чуть ниже будет вариант <<нормальной>> нумерации и советы по её изменению.
% Да, мне так нравится: на первом уровне выравнивание элементов как у обычных абзацев. Проверим теперь вложенные списки.
% \begin{enumerate}
% \item Номера второго уровня.
% \item Номера второго уровня. Проверяем на длииииной-предлиииииииииинной строке, что получается.... Сойдёт.
% \end{enumerate}
% \item По мнению Лукьяненко, человеческий мозг старается подвести любую проблему к выбору
%   из трех вариантов.
% \item Четвёртый (и последний) элемент списка.
% \end{enumerate}

% Теперь мы покажем, как изменить нумерацию на «нормальную», если вам этого захочется. Пара команд в начале документа поможет нам.

% \renewcommand{\labelenumi}{\arabic{enumi})}
% \renewcommand{\labelenumii}{\asbuk{enumii})}

% \begin{enumerate}
% \item Изменим нумерацию на более привычную...
% \item ... нарушим этим гост.
% \begin{enumerate}
% \item Но, пожалуй, так лучше.
% \end{enumerate}
% \end{enumerate}

% В заключение покажем произвольные маркеры в списках. Для них нужен пакет \textbf{enumerate}.
% \begin{enumerate}[1.]
% \item Маркер с арабской цифрой и с точкой.
% \item Маркер с арабской цифрой и с точкой.
% \begin{enumerate}[I.]
% \item Римская цифра с точкой.
% \item Римская цифра с точкой.
% \end{enumerate}
% \end{enumerate}

% В отчётах могут быть и таблицы~--- см. табл.~\ref{tab:tabular} и~\ref{tab:longtable}.
% Небольшая таблица делается при помощи \Code{tabular} внутри \Code{table} (последний
% полностью аналогичен \Code{figure}, но добавляет другую подпись).

% \begin{table}[ht]
%   \caption{Пример короткой таблицы с длинным названием на много длинных-длинных строк}
%   \begin{tabular}{|r|c|c|c|l|}
%   \hline
%   Тело      & $F$ & $V$  & $E$ & $F+V-E-2$ \\
%   \hline
%   Тетраэдр  & 4   & 4    & 6   & 0         \\
%   Куб       & 6   & 8    & 12  & 0         \\
%   Октаэдр   & 8   & 6    & 12  & 0         \\
%   Додекаэдр & 20  & 12   & 30  & 0         \\
%   Икосаэдр  & 12  & 20   & 30  & 0         \\
%   \hline
%   Эйлер     & 666 & 9000 & 42  & $+\infty$ \\
%   \hline
%   \end{tabular}
%   \label{tab:tabular}
% \end{table}

% Для больших таблиц следует использовать пакет \Code{longtable}, позволяющий создавать
% таблицы на несколько страниц по ГОСТ.

% Для того, чтобы длинный текст разбивался на много строк в пределах одной ячейки, надо в
% качестве ее формата задавать \texttt{p} и указывать явно ширину: в мм/дюймах
% (\texttt{110mm}), относительно ширины страницы (\texttt{0.22\textbackslash textwidth})
% и~т.п.

% Можно также использовать уменьшенный шрифт~--- но, пожалуйста, тогда уж во \textbf{всей}
% таблице сразу.

% \begin{center}
%   \begin{longtable}{|p{0.40\textwidth}|c|p{0.30\textwidth}|}
%     \caption{Пример длинной таблицы с длинным названием на много длинных-длинных строк}
%     \label{tab:longtable}
%     \\ \hline
%     Вид шума & Громкость, дБ & Комментарий \\
%     \hline \endfirsthead
%     \subcaption{Продолжение таблицы~\ref{tab:longtable}}
%     \\ \hline \endhead
%     \hline \subcaption{Продолжение на след. стр.}
%     \endfoot
%     \hline \endlastfoot
%     Порог слышимости             & 0     &                                                \\
%     \hline
%     Шепот в тихой библиотеке     & 30    &                                                \\
%     Обычный разговор             & 60-70 &                                                \\
%     Звонок телефона              & 80    & \small{Конечно, это было до эпохи мобильников} \\
%     Уличный шум                  & 85    & \small{(внутри машины)}                        \\
%     Гудок поезда                 & 90    &                                                \\
%     Шум электрички               & 95    &                                                \\
%     \hline
%     Порог здоровой нормы         & 90-95 & \small{Длительное пребывание на более
%     громком шуме может привести к ухудшению слуха}                                        \\
%     \hline
%     Мотоцикл                     & 100   &                                                \\
%     Power Mower                  & 107   & \small{(модель бензокосилки)}                  \\
%     Бензопила                    & 110   & \small{(Doom в целом вреден для здоровья)}     \\
%     Рок-концерт                  & 115   &                                                \\
%     \hline
%     Порог боли                   & 125   & \small{feel the pain}                          \\
%     \hline
%     Клепальный молоток           & 125   & \small{(автор сам не знает, что это)}          \\
%     \hline
%     Порог опасности              & 140   & \small{Даже кратковременное пребывание на
%     шуме большего уровня может привести к необратимым последствиям}                       \\
%     \hline
%     Реактивный двигатель         & 140   &                                                \\
%                                  & 180   & \small{Необратимое полное повреждение
%                                  слуховых органов}                                        \\
%     Самый громкий возможный звук & 194   & \small{Интересно, почему?..}                   \\
%   \end{longtable}
% \end{center}

%%% Local Variables:
%%% mode: latex
%%% TeX-master: "rpz"
%%% End:

\definecolor{gray}{rgb}{0.4,0.4,0.4}
\definecolor{darkblue}{rgb}{0.0,0.0,0.6}
\definecolor{cyan}{rgb}{0.0,0.6,0.6}
\lstloadlanguages{C,C++,csh,Java}

\lstset{
  basicstyle=\ttfamily,
  columns=fullflexible,
  showstringspaces=false,
  commentstyle=\color{gray}\upshape
}

\lstdefinelanguage{XML}
{
  morestring=[b]",
  morestring=[s]{>}{<},
  morecomment=[s]{<?}{?>},
  stringstyle=\color{black},
  identifierstyle=\color{darkblue},
  keywordstyle=\color{cyan},
  morekeywords={xmlns,version,type}% list your attributes here
}

\section{Инфраструктура Ulearn}
\label{cha:design}


В этом разделе описана текущая инфраструктура в образовательной системе Ulearn, а именно
\begin{enumerate}
\item как создаются слайды курса;
\item как происходит их обработка;
\item как осуществляется проверка отправленного студентом решения.
\end{enumerate}


\subsection{Структура курса}
Курс в системе Ulearn состоит из множества слайдов, каждый из которых может быть одним из трех типов:
\begin{enumerate}
\item Lesson — теоретический слайд-урок;
\item Quiz — тест с вопросами разных типов;
\item Exercise — задача на программирование.
\end{enumerate}

Задачи на программирование делятся на 3 типа: 
\begin{enumerate}
\item SingleFileExercise — задача на языке C\# из одного файла, где проверяется, что вывод равен константе;
\item CsProjectExercise — задача на языке C\#;
\item UniversalExercise — задача на языке, который определен конфигурацией docker-образа под эту задачу.
\end{enumerate}

Каждому слайду соответствует xml-файл, который задает тип слайда, информацию, которая будет показана при его отображении и прочие данные. 

Типизация объектов в xml-файле задаётся в соответствующей xsd-схеме.

Рассмотрим несколько конфигурации нескольких типов слайдов.
Первый тип представлен в листинге 1.4.

\begin{lstlisting}[caption={Пример xml-файла lesson-слайда}, language={XML}]
<?xml version='1.0' encoding='UTF-8'?>
<slide xmlns="https://ulearn.me/schema/v2" title="Theory" id="7ac1f6d9-41d6-4309-8eda-8786ccc3f990">
    <markdown>
        // Slide data
    </markdown>
</slide>
\end{lstlisting}

В данной xml-разметке находится главный тег — slide у которого присутствуют обязательные атрибуты: title (название данного слайда) и id (уникальный идентификатор слайда). Внутри тега slide находится тег markdown в который можно размещаться всю информационную нагрузку слайда: текст, ссылки и тд.
\begin{lstlisting}[caption={Пример xml-файла exercise-слайда}, language={XML}]
<?xml version="1.0"?>
<slide.exercise id="717f67cc-f53f-45e3-92f1-8fa20572c6d5" title="Practice" xmlns="https://ulearn.me/schema/v2">
  <scoring group="exercise" passedTestsScore="2" codeReviewScore="8"/>
  <markdown>
    // Task information
  </markdown>
  <exercise.universal exerciseDirName="src/01_fahrenheit" noStudentZip="true">
    <hideSolutions>true</hideSolutions>
    <hideExpectedOutput>true</hideExpectedOutput>
    <userCodeFile>task.js</userCodeFile>
    <includePathForChecker>../TestsRunner</includePathForChecker>
    <dockerImageName>js-sandbox</dockerImageName>
    <run>node docker-test-runner.js</run>
    <region>Task</region>
  </exercise.universal>
</slide.exercise>
\end{lstlisting}

В разметке слайда типа exercise, который представлен в листинге 1.5, различия начинаются с главного тега: вместо slide в данном типе находится slide.exercise. \\ Внутри slide.exercise, помимо уже известного markdown, присутствуют блоки: scoring — тег, отвечающий за конфигурацию расчета баллов для этой задачи, exercise.universal — тег, вместе со своими аттрибутами и вложенными тегами, отвечают за конфигурацию самого задания, механизма проверки и тд.

\subsection{Обработка конфигурации слайдов}

За обработку конфигураций слайдов ответственны сущности реализующие следующий контракт, представленный на листинге 1.6:

\begin{lstlisting}[caption={Контракт конфигуратора слайдов}]
public interface ISlide
	{
		Guid Id { get; }
		string Title { get; }
		SlideBlock[] Blocks { get; }
		
		void Validate(SlideLoadingContext context);
		void BuildUp(SlideLoadingContext context);
	}
\end{lstlisting}

Метод Validate отвечают за проверку корректности заполнения конфигурации.
А BuildUp — за логику построения по данным конфигурации html-разметки.


\subsection{Обработка конфигурации блоков}
За обработку конфигураций блоков внутри слайда ответственны сущности, которые наследуются абстрактного класса SlideBlock. Главные пункты этого класса соответствуют конфигураторам слайдов: методы Validate и BuildUp.

\subsection{Проверка решения студента}

После того как пользователь ввел своё решение в редактор сервиса и нажал кнопку "Отправить"     вызывается метод RunSolution контроллера ExerciseController, в который передаётся информация о курсе, слайде и отправленном решении. Его структура описана в листинге 1.7.

\begin{lstlisting}[language={C}, caption={Сигнатура метода RunSolution}]
public async Task<ActionResult<RunSolutionResponse>> RunSolution(
			[FromRoute] Course course, 
			[FromRoute] Guid slideId,
			[FromBody] RunSolutionParameters parameters,
			[FromQuery] Language language)
			
\end{lstlisting}

Данный метод осуществляет некоторые первичные проверки отправленного кода: на размер, на то, что код этой задачи уже проверяется у текущего пользователя и тд. А после этого ставит ответ студента в очередь на проверку.

Сервис RunCheckerJob с определенной частотой проверяет очередь отправленных решений. Если очередь непустая, то сервис забирает оттуда решение и запускает для него проверку в docker-контейнере. Во время запуска отправляет внутрь контейнера всё необходимое для проверки задачи: чеккер, решение студента и прочие файлы указанные при настройке задачи в xml-файле.

В Листинге 1.8 представлен пример конфигурирования docker-образа. 
Стоит обратить внимание на создание внутри образа нового пользователя student. Данный подход пригодится в дальнейшем, когда встанет задача конфигурирования образа и важно будет учесть права на чтения тестовых данных в режиме студента. 
\begin{lstlisting}[caption={Dockerfile для проверки задач по курсу JavaScript}]
FROM node:14-alpine

RUN apk update && apk upgrade

# Add new user for sandbox
RUN addgroup -S student && adduser -S -g student student \
    && chown -R student:student /home/student

COPY ./app/ /app/

RUN mkdir -p /source \
    && chown -R student:student /app

WORKDIR app

# Run user as non privileged.
USER student

# Install deps for tests.
RUN npm ci
\end{lstlisting}

Стоит заметить, что все чеккеры, необходимые для проверки решения студента должны быть написаны самим составителем курса и они будут специфичны для конкретного языка из-за зависимости от фреймворка тестирования.

После проверки из docker-контейнера возвращается результат в RunCheckerJob по которому мы можем узнать статус решения.

Бывают статусы, представленные в листинге 1.9:

\begin{lstlisting}[caption={Модель описывающая статус решения}]
public enum Verdict
	{
		Ok = 1,
		CompilationError = 2,
		RuntimeError = 3,
		SecurityException = 4,
		SandboxError = 5,
		OutputLimit = 6,
		TimeLimit = 7,
		MemoryLimit = 8,
		WrongAnswer = 9
	}
\end{lstlisting}

Результат сохраняется в базу данных и отображается пользователю.			
			
% \paragraph{Проверка} параграфа. Вроде работает.
% \paragraph{Вторая проверка} параграфа. Опять работает.

% Вот.

% \begin{itemize}
% \item Это список с <<палочками>>.
% \item Хотя он и не по ГОСТ, кажется.
% \end{itemize}

% \begin{enumerate}
% \item Поэтому для списка, начинающегося с заглавной буквы, лучше список с цифрами.
% \end{enumerate}

% Формула \ref{F:F1} совершено бессмысленна.

% %Кстати, при каких-то условиях <<удавалось>> получить двойный скобки вокруг номеров формул. Вопрос исследуется.

% \begin{equation}
% a= cb
% \label{F:F1}
% \end{equation}


% Окружение \texttt{cases} опять работает (см. \ref{F:F2}), спасибо И. Короткову за исправления..


% \begin{equation}
% a= \begin{cases}
%  3x + 5y + z, \mbox{если хорошо} \\
%  7x - 2y + 4z, \mbox{если плохо}\\
%  -6x + 3y + 2z, \mbox{если совсем плохо}
% \end{cases}
% \label{F:F2}
% \end{equation}

% \section{Подсистема всякой ерунды}

% Культурная вставка dot-файлов через утилиту dot2tex (рис.~\ref{fig:fig02}).

% \begin{figure}
%   \centering
% % [width=0.5\textwidth] --- регулировка ширины картинки
%   \includegraphics{figures/pic01}
%   \caption{Рисунок}
%   \label{fig:fig02}
% \end{figure}


% \subsection{Блок-схема всякой ерунды}

% \subsubsection*{Кстати о заголовках}

% У нас есть и \Code{subsubsection}. Только лучше её не нумеровать.

%%% Local Variables:
%%% mode: latex
%%% TeX-master: "rpz"
%%% End:

\definecolor{gray}{rgb}{0.4,0.4,0.4}
\definecolor{darkblue}{rgb}{0.0,0.0,0.6}
\definecolor{cyan}{rgb}{0.0,0.6,0.6}
\lstloadlanguages{C,C++,csh,Java}

\definecolor{red}{rgb}{0.6,0,0} 
\definecolor{blue}{rgb}{0,0,0.6}
\definecolor{green}{rgb}{0,0.8,0}
\definecolor{cyan}{rgb}{0.0,0.6,0.6}

\lstset{
  basicstyle=\ttfamily,
  columns=fullflexible,
  showstringspaces=false,
  commentstyle=\color{gray}\upshape
}

\lstdefinelanguage{XML}
{
  morestring=[b]",
  morestring=[s]{>}{<},
  morecomment=[s]{<?}{?>},
  stringstyle=\color{black},
  identifierstyle=\color{darkblue},
  keywordstyle=\color{cyan},
  morekeywords={xmlns,version,type}% list your attributes here
}



\section{Проектирование решения}
\label{cha:design}


В этом разделе спроектировано решение в рамках принятых в системе Ulearn паттернов.
\\


Исходя из рассмотренной структуры в пункте 1.2.1 становится ясно, что задачи должны находиться на слайде типа Exercise. Дальнейшие шаги имеют несколько вариантов реализации. Рассмотрим их и выделим плюсы и минусы каждого.

\begin{enumerate}
\item Создать новый тип задач, который будет независим от уже существующих в системе. Прописать его типизацию в xsd-схеме. Создать сущность для обработки конфигурации, реализующую контракт ISlide. А так же реализовать систему проверки;
\\ \textbf{Плюсы:}
     \begin{enumerate}                  
         \item Автору при заполнении xml-файла не будет подсказываться ничего лишнего, относящегося к другим типам задач.
    \end{enumerate}
\\ \textbf{Минусы:}
    \begin{enumerate}
        \item Нарушение принципа DRY, так как большую часть логики можно было переиспользовать из уже реализованной в системе.
    \end{enumerate}
\item Внести логику реализуемых задач внутрь сущностей типа UniversalExercise;
\\ \textbf{Плюсы:}
     \begin{enumerate}                  
         \item Логика системы переиспользуется;
         \item сокращается количество кода, которое необходимо написать.
    \end{enumerate}
\\ \textbf{Минусы:}
    \begin{enumerate}
        \item Нарушение принципа SRP, так как в нашей конфигурации будет поля, которые не нужны в UniversalExercise и наоборот.
    \end{enumerate}
    
\item Создать новый тип задач, сделать его наследником типа UniversalExercise и в xsd-схеме и в обработке конфигурации (так как остальные специализированы только на проверке задач, реализованных на языке C\#);
\\ \textbf{Плюсы:}
     \begin{enumerate}                  
         \item Логика системы переиспользуется;
         \item сокращается количество кода, которое необходимо написать.
    \end{enumerate}
\\ \textbf{Минусы:}
    \begin{enumerate}
        \item Автор при создании слайда в xml будет видеть в подсказки, которые не участвуют в конфигурации нашей задачи.
    \end{enumerate}
\end{enumerate}

Проанализировав данные варианты было принято решение совместить вариант а) и вариант в). В результате получается решение: \\
Создать новый тип задач, сделать его наследником типа UniversalExercise в обработке конфигурации, но прописать для него независимую типизацию в xsd-схеме. Таким образом единственный минус варианта в) будет решен.			
			
% \paragraph{Проверка} параграфа. Вроде работает.
% \paragraph{Вторая проверка} параграфа. Опять работает.

% Вот.

% \begin{itemize}
% \item Это список с <<палочками>>.
% \item Хотя он и не по ГОСТ, кажется.
% \end{itemize}

% \begin{enumerate}
% \item Поэтому для списка, начинающегося с заглавной буквы, лучше список с цифрами.
% \end{enumerate}

% Формула \ref{F:F1} совершено бессмысленна.

% %Кстати, при каких-то условиях <<удавалось>> получить двойный скобки вокруг номеров формул. Вопрос исследуется.

% \begin{equation}
% a= cb
% \label{F:F1}
% \end{equation}


% Окружение \texttt{cases} опять работает (см. \ref{F:F2}), спасибо И. Короткову за исправления..


% \begin{equation}
% a= \begin{cases}
%  3x + 5y + z, \mbox{если хорошо} \\
%  7x - 2y + 4z, \mbox{если плохо}\\
%  -6x + 3y + 2z, \mbox{если совсем плохо}
% \end{cases}
% \label{F:F2}
% \end{equation}

% \section{Подсистема всякой ерунды}

% Культурная вставка dot-файлов через утилиту dot2tex (рис.~\ref{fig:fig02}).

% \begin{figure}
%   \centering
% % [width=0.5\textwidth] --- регулировка ширины картинки
%   \includegraphics{figures/pic01}
%   \caption{Рисунок}
%   \label{fig:fig02}
% \end{figure}


% \subsection{Блок-схема всякой ерунды}

% \subsubsection*{Кстати о заголовках}

% У нас есть и \Code{subsubsection}. Только лучше её не нумеровать.

%%% Local Variables:
%%% mode: latex
%%% TeX-master: "rpz"
%%% End:

\definecolor{gray}{rgb}{0.4,0.4,0.4}
\definecolor{darkblue}{rgb}{0.0,0.0,0.6}
\definecolor{cyan}{rgb}{0.0,0.6,0.6}
\lstloadlanguages{C,C++,csh,Java}

\definecolor{red}{rgb}{0.6,0,0} 
\definecolor{blue}{rgb}{0,0,0.6}
\definecolor{green}{rgb}{0,0.8,0}
\definecolor{cyan}{rgb}{0.0,0.6,0.6}

\lstset{
  basicstyle=\ttfamily,
  columns=fullflexible,
  showstringspaces=false,
  commentstyle=\color{gray}\upshape
}

\lstdefinelanguage{XML}
{
  morestring=[b]",
  morestring=[s]{>}{<},
  morecomment=[s]{<?}{?>},
  stringstyle=\color{black},
  identifierstyle=\color{darkblue},
  keywordstyle=\color{cyan},
  morekeywords={xmlns,version,type}% list your attributes here
}


\section{Техническая реализация}
\label{cha:impl}

В этом разделе описана техническая часть работы, реализованная в рамках спроектированных в прошлом пункте идей.


\subsection{Xsd-схема блока задачи}

Как можно увидеть в Листинге 1.10 типизация тега slide.polygon не наследуется ни от каких других типов, а значит автору будет подсказываться только то, что действительно нужно.

\begin{lstlisting}[caption={Реализация xsd-схемы тега slide.polygon}, language={XML}]
<xs:element name="slide.polygon" type="PolygonSlide" />
<xs:complexType name="PolygonSlide">
	<xs:all>
		<xs:element name="scoring"type="ExerciseScoring" minOccurs="0"/>
		<xs:element name="polygonPath"type="xs:string"/>
		<xs:element name="markdown"type="MarkdownBlock" minOccurs="0"/>
		<xs:element name="exercise.polygon"type="PolygonExercise" />
	</xs:all>
	<xs:attribute name="id" type="NonEmptyString"use="required" />
	<xs:attribute name="hide" type="xs:boolean" />
</xs:complexType>
\end{lstlisting}

\subsection{Обработка конфигурации слайда}

За обработку конфигурации слайдов с задачами отвечает сущность ExerciseSlide. Если задачи типа UniversalExercise, то внутри ExerciseSlide находится блок  поэтому необходимо наш конфигуратор PolygonExerciseSlide отнаследовать от него.


Первая задача, которую нужно решить для генерации слайда — получить доступ к данным о задаче. Для этого вынесем свойство нашей сущности путь до директории с этой информацией.

На этапе валидации слайда необходимо знать, что автор курса не забыл про это поле и указал его. Поэтому оно проверяется явно. Это представлено на листинге 1.11.

\begin{lstlisting}[caption={Проверка полей}]
    [XmlElement("polygonPath")]
    public string PolygonPath { get; set; }
    public override void Validate(SlideLoadingContext context)
		{
			if (string.IsNullOrEmpty(PolygonPath))
				throw new CourseLoadingException("polygonPath not found in slide.polygon");
			base.Validate(context);
		}
\end{lstlisting}

Полный текст задачи генерируется автоматически благодаря вложенной в архив задачи информации. В рассмотренном листинге 1.1 показано, что существует файлы problem.html и problem-statements.css, которые находятся по пути /statements/.html/russian/problem.html и /statements/.html/russian/problem-statements.html соответственно. Они определяют визуальное представление задачи, которое состоит из легенды, примеров входных и выходных данных, информации об ограничениях задачи.
Эти файлы используются в нашей обработке конфигурации: файлы читаются и их содержимое подставляется в html-код, выводящийся пользователю.

Оставлена возможность для автора курса на своё усмотрение переопределить выводящуюся пользователю информацию. Для этого ему необходимо внутри слайда добавить тег markdown с необходимыми данными и он будет рассматриваться приемущественнее, чем текст в архиве задачи.

Так же если в архиве задачи есть визуальное представление в виде pdf, то пользователю дается ссылка на этот файл.\\

Эта логика представлена в листинге 1.12.
\begin{lstlisting}[caption={Формирование блоков}]
private IEnumerable<SlideBlock> GetBlocksProblem(
        string statementsPath, 
        string courseId, 
        Guid slideId)
		{
			var markdownBlock = Blocks.FirstOrDefault(block => block is MarkdownBlock);
			if (markdownBlock != null)
			{
				yield return markdownBlock;
			}
			else if(Directory.Exists(Path.Combine(statementsPath, ".html")))
			{
				var htmlDirectoryPath = Path.Combine(statementsPath, ".html", "russian");
				var htmlData = File.ReadAllText(
				Path.Combine(htmlDirectoryPath, "problem.html"));
				yield return RenderFromHtml(htmlData);
			}

			var pdfLink = PolygonPath + "/statements/.pdf/russian/problem.pdf";
			yield return new MarkdownBlock($"[Download problem conditions in PDF format]({pdfLink})");
		}
\end{lstlisting}

Далее описана непосредственно обработка конфигурации в методе BuildUp. Рассмотрим листинг 1.13. 

\begin{lstlisting}[caption={Обработка конфигурации в PolygonExerciseSlide}]
public override void BuildUp(SlideLoadingContext context)
{
	var statementsPath = Path.Combine(context.Unit.Directory.FullName, PolygonPath, "statements");
			
	Blocks = GetBlocksProblem(statementsPath, context.CourseId, Id)
			.Concat(Blocks.Where(block => !(block is MarkdownBlock)))
			.ToArray();
			
	var problem = GetProblem(
		Path.Combine(context.Unit.Directory.FullName, PolygonPath, "problem.xml"), 
		context.CourseSettings.DefaultLanguage
	);

	var polygonExercise = Blocks.Single(block => block is PolygonExerciseBlock) as PolygonExerciseBlock;
			
	polygonExercise!.ExerciseDirPath = Path.Combine(PolygonPath);
	polygonExercise.TimeLimitPerTest = problem.TimeLimit;
	polygonExercise.TimeLimit = (int)Math.Ceiling(problem.TimeLimit * problem.TestCount);
	polygonExercise.UserCodeFilePath = problem.PathAuthorSolution;
	polygonExercise.Language = LanguageHelpers.GuessByExtension(new FileInfo(polygonExercise.UserCodeFilePath));
	polygonExercise.DefaultLanguage = context.CourseSettings.DefaultLanguage;
	polygonExercise.RunCommand = $"python3.8 main.py {polygonExercise.Language} {polygonExercise.TimeLimitPerTest} {polygonExercise.UserCodeFilePath.Split('/', '\\')[1]}";
			
	Title = problem.Title;
			
	PrepareSolution(
    	Path.Combine(context.Unit.Directory.FullName, PolygonPath, polygonExercise.UserCodeFilePath)
	);

	base.BuildUp(context);
}

\end{lstlisting}

В строке 3 формируется путь до каталога statements.

В строках 5-7, используя уже рассмотренный метод, генерируются блоки представления задачи.

В строках 9-12 вызывается метод, который из файла problem.xml получает информацию о задаче: ограничение по времени на тест, количество тестов, название задачи, путь до авторского решения. 

В строке 14 из множества блоков данного слайда получаю конфигуратор блока PolygonExercise и в строках 16-22 происходит его инициализация:
\begin{enumerate}
    \item пути до директории с данными задачи;
    \item ограничении времени на тест и на выполнение всех тестов;
    \item пути до авторского решения;
    \item языка авторского решения и языка по-умолчанию(язык, который будет предложен пользователю курса в первую очередь);
    \item команду для запуска проверяющей программы внутри docker-контейнера.
\end{enumerate}

В строках 26-28 происходит вызов метода PrepareSolution, задача которого обернуть авторского решение в region, как это представлено в листинге 1.14. Внутрь этого блока будет подставляться решение студента при отправке.
\begin{lstlisting}[caption={Оборачивание кода в region}]
\\region Task
    Code
\\endregion Task
\end{lstlisting}

Завершает метод вызов base.BuildUp, который конфигурируют всю остальную логику связанную с формированием слайдов.

\subsection{Обработка конфигурации блока}

За обработку конфигурации блока, как уже было сказано в прошлом пункте, отвечает класс PolygonExerciseBlock, который является потомком UniversalExerciseBlock.

В данном классе расположена информация о языках, проверка на которых доступна студенту. Отрывок инициализации представлен в листинге 1.15.

\begin{lstlisting}[caption={Отрывок инициализации LanguagesInfo}]
public static Dictionary<Language, LanguageLaunchInfo> LanguagesInfo = new Dictionary<Language, LanguageLaunchInfo>
		{
			[Common.Language.Java] = new LanguageLaunchInfo
			{
				Compiler = "Java 14",
				CompileCommand = "javac {source}",
				RunCommand = "java {compilation-result-file}"
			},
			...
		}
\end{lstlisting}

Наибольший интерес в данном классе представляют метод создания RunnerSubmission — сущности содержащую в себе информацию об отправке студентом кода на проверку и способе его запуска. Единственно, что входит в ответственность PolygonExerciseBlock в этой задаче — формирование команды запуска. Всё остальную логику выполняет UniversalExerciseBlock. Данная логика описана в листинге 1.16.

\begin{lstlisting}[caption={Формирование RunnerSubmission}]
public RunnerSubmission CreateSubmission(string submissionId, string code, Language language)
{
	var submission = 
	    base.CreateSubmission(submissionId, code);
	if (!(submission is CommandRunnerSubmission commandRunnerSubmission))
		return submission;
	
	commandRunnerSubmission.RunCommand = RunCommandWithArguments(language);
	
	return commandRunnerSubmission;
}

private string RunCommandWithArguments(Language language)
{
	return $"python3.8 main.py {language} {TimeLimitPerTest} {UserCodeFilePath.Split('/', '\\')[1]}";
}
\end{lstlisting}

Внутри Docker-контейнера будет выполняться команда, которая находится в свойстве RunCommand. Она запустит python-код и передаст ему в аргументы язык, на котором написано решение, ограничение по времени на тест, а так же путь до файла с решением.

\subsection{Проверка решения студента}

После отправки решения студентом, запускается docker-контейнер в который подкладывается отправленное решение и начинает выполнение проверяющая программа. 

Рассмотрим конфигурацию из Dockerfile, описанную в листинге 1.17.

\begin{lstlisting}[caption={Конфигурация docker-образа}]
FROM ubuntu:20.04

ENV NVM_DIR /usr/local/nvm
ENV NODE_VERSION 14.15.3
ARG DEBIAN_FRONTEND=noninteractive
ENV TZ=Europe/Ekaterinburg

RUN apt-get update 
&& apt-get -y upgrade
&& apt-get install -y wget 
&& apt-get -y install curl \
# Python 3.8
&& apt-get -y install python3.8 \ 
# C, C++
&& apt-get -y install build-essential \
# C\#
&& wget https://packages.microsoft.com/config
    /ubuntu/20.10/packages-microsoft-prod.deb -O packages-microsoft-prod.deb \
&& dpkg -i packages-microsoft-prod.deb \
&& apt-get update && apt-get install -y apt-transport-https && apt-get install -y dotnet-sdk-5.0 \
&& apt-get install -y apt-transport-https && apt-get install -y dotnet-runtime-5.0
# Java
RUN apt-get install -y openjdk-14-jdk \
# JavaScript
&& curl --silent -o- https://raw.githubusercontent.com
    /creationix/nvm/v0.31.2/install.sh | bash \
&& . $NVM_DIR/nvm.sh \
&& nvm install $NODE_VERSION \
&& nvm alias default $NODE_VERSION \
&& nvm use default

ENV NODE_PATH $NVM_DIR/v$NODE_VERSION/lib/node_modules
ENV PATH $NVM_DIR/versions/node/v$NODE_VERSION/bin:$PATH
ENV NODE_REPL_HISTORY ''

COPY ./app/ /app/

WORKDIR app

RUN useradd student && chmod 700 /app/tests
\end{lstlisting}

Комментариями в коде помечены языки, которые становятся доступны после выполнения этих команд. 

Обратим внимание на строку 39, в которой создается новый пользователь и устанавливаются права доступа на директорию с тестами. Таким образом обеспечивается невозможность получения тестовых данных из решения.

Далее рассмотрим механизм проверки решения. 

Первоочередная задача, которая возникает в процессе выполнения проверя — определить язык решения и команды для компиляции и выполнения исходного кода.

Для инкапсуляции данных об этих командах был создан контракт ISourceCodeRunInfo. Рассмотрим листинге 1.18.
\begin{lstlisting}[caption={Контракт ISourceCodeRunInfo}]
class ISourceCodeRunInfo:
    __metaclass__ = ABCMeta
    @abstractmethod
    def file_extension(self): pass
    @abstractmethod
    def format_build_command(self, 
        code_filename: str, 
        result_filename: str) -> str: pass
    @abstractmethod
    def need_build(self) -> bool: pass
    @abstractmethod
    def format_run_command(self, filename: str) -> str: pass
\end{lstlisting}

Метод \texttt{file\_extension} возвращает расширение файла с исходным кодом. \\
Метод \texttt{format\_build\_command} выполняет задачу формирования команды для компилирования исходного кода. В случае если исходный код некомпилируемый, например, исходных код на языке Python, то возвращается None.\\
Метод \texttt{format\_run\_command} возвращает команду для запуска решения.

Приведу пример реализации данной сущность для языка C++ в листинге 1.19.

\begin{lstlisting}[caption={Реализация ISourceCodeRunInfo для языка C++}]
class CppRunInfo(ISourceCodeRunInfo):
    def file_extension(self):
        return '.cpp'
    def format_run_command(self, filename: str) -> str:
        return f'./{filename}'
    def need_build(self) -> bool:
        return True
    def format_build_command(self, code_filename: str, result_filename: str) -> str:
        return f'g++ -o {result_filename} -O2 {code_filename}'
\end{lstlisting}

Для поиска нужной реализации данного контракта по названию языка была написана функция \texttt{get\_run\_info\_by\_language\_name}, она представлена в листинге 1.20.
\begin{lstlisting}[caption={Функция поиска по названию языка реализации ISourceCodeRunInfo}]
def get_run_info_by_language_name(language_name: str) -> ISourceCodeRunInfo:
    if language_name == 'cpp':
        return CppRunInfo()
    elif language_name == 'python3':
        return PythonRunInfo()
    ...
\end{lstlisting}

В листинге 1.21 представленн класс TaskCodeRunner, который реализует логику сборку, запуск кода и запуск теста, используя взаимодействие с процессами. 

\begin{lstlisting}[caption={Сигнатура класса TaskCodeRunner}]
class TaskCodeRunner:
    def build(self, code_filename: str, result_filename: str): pass
    def run(self, test_file: str): pass
    def run_test(self, code_filename: str, test_file: str): pass
\end{lstlisting}

Проверка решения студента осуществляется по следующему алгоритму:
\begin{enumerate}
    \item сборка исходного кода чеккера check.cpp, который мы получили из директории задачи;
    \item предобработка решения студента: удаления строк определения региона; в случае с java кодом переименование названия файла в соответствие с названием класса внутри файла;
    \item получение ISourceCodeRunInfo по названию языка полученному в аргументах программы;
    \item компиляция исходного кода студента, если это требуется
    \item для каждого теста
    \begin{enumerate}
        \item запуск программы студента на этом тесте
        \item сравнение результата выполнения с авторским решением используя чеккер
    \end{enumerate}
    \item обработка ошибок, при возникновении
\end{enumerate}

Результатом выполнения проверяющей программы является json, структура которого продемонстрирована в листинге 1.22.

\begin{lstlisting}[caption={RunningResult}]
{
    "Verdict": "Ok" | "WrongAnswer" | "TimeLimit" | ...
    "TestResultInfo": {
            "TestNumber": "1" | "2" | ... ,
            "Input": // test data,
            "CorrectOutput": // correct test output ,
            "StudentOutput": // student output
        }
}
\end{lstlisting}

\subsection{Логирование}

Помимо этого, у авторов курсов могут возникнуть вопросы почему решение студента не проходит определенные тесты или подозрение в том, что система работает неправильно. Поэтому было реализовано логирование жизненного цикла проверяющей системы с выводом в интерфейс, но только для администраторов данного курса. 

\subsection{Добавление нового языка}
В дальнейшем у авторов может возникнуть желание добавить новый язык программирования, чтобы у студентов был еще более разнообразный выбор. Для встраивания еще одного языка следует выполнить следующие шаги для этого языка: 

\begin{enumerate}
    \item прописать команды установки окружния в dockerfile;
    \item реализовать сущность ISourceCodeRunInfo;
    \item прописать условие для обнаружения ISourceCodeRunInfo по названию языка в методе \texttt{get\_run\_info\_by\_language\_name};
    \item в классе PolygonExerciseBlock прописать информацию о языке в поле LanguagesInfo.
\end{enumerate}

\subsection{Тестирование}

Логика, реализованная в рамках данной работы, была протестирована несколькими методами. 

\begin{enumerate}
    \item логика работы docker-образа была проверена unit-тестом, который запускает его, отправляет туда тестовую задачу с авторским решением и проверяет, что результатом выполнения стал статус Ok;
    \item курс для тестирования системы Ulearn был дополнен несколькими слайдами, где заложена возможность ручного тестирования данной логики.
\end{enumerate}




% В дfileанном разделе описано изготовление и требование всячины. Кстати,
% в Latex нужно эскейпить подчёркивание (писать <<\verb|some\_function|>> для \Code{some\_function}).

% \ifPDFTeX
% Для вставки кода есть пакет \Code{listings}. К сожалению, пакет \Code{listings} всё ещё
% работает криво при появлении в листинге русских букв и кодировке исходников utf-8.
% В данном примере он (увы) на лету конвертируется в koi-8 в ходе сборки pdf.

% Есть альтернатива \Code{listingsutf8}, однако она работает лишь с
% \Code{\textbackslash{}lstinputlisting}, но не с окружением \Code{\textbackslash{}lstlisting}

% Вот так можно вставлять псевдокод (питоноподобный язык определен в \Code{listings.inc.tex}):

% \begin{lstlisting}[style=pseudocode,caption={Алгоритм оценки дипломных работ}]
% def EvaluateDiplomas():
%     for each student in Masters:
%         student.Mark := 5
%     for each student in Engineers:
%         if Good(student):
%             student.Mark := 5
%         else:
%             student.Mark := 4
% \end{lstlisting}

% Еще в шаблоне определен псевдоязык для BNF:

% \begin{lstlisting}[style=grammar,basicstyle=\small,caption={Грамматика}]
%   ifstmt -> "if" "(" expression ")" stmt |
%             "if" "(" expression ")" stmt1 "else" stmt2
%   number -> digit digit*
% \end{lstlisting}

% В листинге~\ref{lst:sample01} работают русские буквы. Сильная магия. Однако, работает
% только во включаемых файлах, прямо в \TeX{} нельзя.

% % Обратите внимание, что включается не ../src/..., а inc/src/...
% % В Makefile есть соответствующее правило для inc/src/*,
% % которое копирует исходные файлы из ../src и конвертирует из UTF-8 в KOI8-R.
% % Кстати, поэтому использовать можно только русские буквы и ASCII,
% % весь остальной UTF-8 вроде CJK и египетских иероглифов -- нельзя.

% \lstinputlisting[language=C,caption=Пример (\Code{test.c}),label=lst:sample01]{listings/test.c}

% \else

% Для вставки кода есть пакет \texttt{minted}. Он хорош всем кроме: необходимости Python (есть во всех нормальных (нет, Windows, я не про тебя) ОС) и Pygments и того, что нормально работает лишь в \XeLaTeX.

% Можно пользоваться расширенным BFN:

% \begin{listing}[H]
% \begin{ebnfcode}
%  letter = "A" | "B" | "C" | "D" | "E" | "F" | "G"
%       | "H" | "I" | "J" | "K" | "L" | "M" | "N"
%       | "O" | "P" | "Q" | "R" | "S" | "T" | "U"
%       | "V" | "W" | "X" | "Y" | "Z" ;
% digit = "0" | "1" | "2" | "3" | "4" | "5" | "6" | "7" | "8" | "9" ;
% symbol = "[" | "]" | "{" | "}" | "(" | ")" | "<" | ">"
%       | "'" | '"' | "=" | "|" | "." | "," | ";" ;
% character = letter | digit | symbol | "_" ;
 
% identifier = letter , { letter | digit | "_" } ;
% terminal = "'" , character , { character } , "'" 
%          | '"' , character , { character } , '"' ;
 
% lhs = identifier ;
% rhs = identifier
%      | terminal
%      | "[" , rhs , "]"
%      | "{" , rhs , "}"
%      | "(" , rhs , ")"
%      | rhs , "|" , rhs
%      | rhs , "," , rhs ;
 
% rule = lhs , "=" , rhs , ";" ;
% grammar = { rule } ;
% \end{ebnfcode}
% \caption{EBNF определённый через EBNF}
% \label{lst:ebnf}
% \end{listing}

% А вот в листинге \ref{lst:c} на языке C работают русские комменты. Спасибо Pygments и Minted за это.

% \begin{listing}[H]
% \cfile{inc/src/test.c}
% \caption{Пример — test.c} 
% \end{listing}
% \label{lst:c}

% \fi

% % Для вставки реального кода лучше использовать \texttt{\textbackslash lstinputlisting} (который понимает
% % UTF8) и стили \Code{realcode} либо \Code{simplecode} (в зависимости от размера куска).




% Можно также использовать окружение \Code{verbatim}, если \Code{listings} чем-то не
% устраивает. Только следует помнить, что табы в нём <<съедаются>>. Существует так же команда \Code{\textbackslash{}verbatiminput} для вставки файла.

% \begin{verbatim}
% a_b = a + b; // русский комментарий
% if (a_b > 0)
%     a_b = 0;
% \end{verbatim}

%%% Local Variables:
%%% mode: latex
%%% TeX-master: "rpz"
%%% End:


\backmatter %% Здесь заканчивается нумерованная часть документа и начинаются ссылки и
            %% заключение

\Conclusion % заключение к отчёту

В рамках работы была поставлена следующая цель: реализовать поддержку задач в формате Codeforces в образовательной платформе Ulearn, а именно возможность размещения таких курсов, автопроверки решений по задачам.

В работе был проведен анализ предметной области, а именно создания задачи формата Codeforces, описаны пункты, за которые будет ответственна система и пункты, которые остаются на ответственность авторов курсов. Была изучена текущая инфраструктура системы Ulearn и рассмотрены варианты её расширения для текущей задачи. Из этих вариантов был выделен один, который не имел противоречий с принципами ООП и с установленными в системе Ulearn правилами. 

Данное решение был реализовано, протестировано методами ручного и unit тестирования и выпущено в релиз.

После релиза реализовывались пожелания от авторов курса. Например, возможность вывода информации о тесте(входные данные, правильные результат, результат решения) при неправильном результате решения студента. 

Также аосле релиза благодаря новым возможностям был создан курс по изучению языка программирования Python для школьников.

В дальнейших планах присутствует создание курса по алгоритмам и структурам данных для студентов УрФУ.



%%% Local Variables: 
%%% mode: latex
%%% TeX-master: "rpz"
%%% End: 

%\include{conclusion} было для преддипломной практики
% % Список литературы при помощи BibTeX
% Юзать так:
%
% pdflatex rpz
% bibtex rpz
% pdflatex rpz

\bibliographystyle{gost780u}
% \bibliography{rpz}

\begin{thebibliography}{9}

\bibitem{}
\textit{Система подготовки олимпиадных задач по программированию в УрГУ}
Алексей Самсонов, Александр Ипатов\\
URL: \texttt{https://sp.urfu.ru/library/system.html} (дата обращения: 15.10.2020).

\bibitem{}
\textit{E-learning platform by SKB Kontur.}
kontur-edu.
URL: \texttt{https://github.com/kontur-edu/Ulearn}
SKB Kontur. 

\bibitem{}
\textit{Коротко о testlib.h.}
Mike Mirzayanov.
URL: \texttt{https://codeforces.com/testlib} (дата обращения: 16.10.2020)

\bibitem{}
\textit{Валидаторы на testlib.h.}
 I\_love\_Hoang\_Yen.
URL: \texttt{https://codeforces.com/blog/entry/18426} (дата обращения: 16.10.2020)
 
 \bibitem{}
 \textit{Генераторы на testlib.h.}
 Mike Mirzayanov.
 URL: \texttt{https://codeforces.com/blog/entry/18291} (дата обращения: 16.10.2020)
 
 \bibitem{}
 \textit{Checkers with testlib.h.}
 Zlobober.
URL: \texttt{https://codeforces.com/blog/entry/18431} (дата обращения: 16.10.2020)
 
 \bibitem{}
Mike Mirzayanov.
URL: \texttt{https://polygon.codeforces.com/} (дата обращения: 16.10.2020)



\end{thebibliography}

% [1] Nisheeth Shrivastava, Chiranjeeb Buragohain, Divyakant Agrawal,
% Subhash Suri Medians and Beyond: New Aggregation Techniques
% for Sensor Networks

%%% Local Variables: 
%%% mode: latex
%%% TeX-master: "rpz"
%%% End: 


% \appendix   % Тут идут приложения

% \include{91-appendix2}

\end{document}

%%% Local Variables:
%%% mode: latex
%%% TeX-master: t
%%% End:
