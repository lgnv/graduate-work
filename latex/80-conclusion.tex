\Conclusion % заключение к отчёту

В рамках работы была поставлена следующая цель: реализовать поддержку задач в формате Codeforces в образовательной платформе Ulearn, а именно возможность размещения таких курсов, автопроверки решений по задачам.

В работе был проведен анализ предметной области, а именно создания задачи формата Codeforces, описаны пункты, за которые будет ответственна система и пункты, которые остаются на ответственность авторов курсов. Была изучена текущая инфраструктура системы Ulearn и рассмотрены варианты её расширения для текущей задачи. Из этих вариантов был выделен один, который не имел противоречий с принципами ООП и с установленными в системе Ulearn правилами. 

Данное решение был реализовано, протестировано методами ручного и unit тестирования и выпущено в релиз.

После релиза реализовывались пожелания от авторов курса. Например, возможность вывода информации о тесте(входные данные, правильные результат, результат решения) при неправильном результате решения студента. 

Также аосле релиза благодаря новым возможностям был создан курс по изучению языка программирования Python для школьников.

В дальнейших планах присутствует создание курса по алгоритмам и структурам данных для студентов УрФУ.



%%% Local Variables: 
%%% mode: latex
%%% TeX-master: "rpz"
%%% End: 
