% Также можно использовать \Referat, как в оригинале
\begin{abstract}
Выпускная квалификационная работа состоит из введения, 4 разделов и 16 подразделов, заключения, списка используемых источников состоящего из 7 источников. Работа изложена на 36 листах печатного текста, содержит 22 листинга.

Ключевые слова: POLYGON, TIMUS, CODEFORCES, ОБРАЗОВАТЕЛЬНЫЕ КУРСЫ, ULEARN, DOCKER, ОЛИМПИАДНЫЕ ЗАДАЧИ, АВТОПРОВЕРКА.

Целью дипломной работы является реализация поддержки задач в формате Codeforces в образовательной платформе Ulearn: возможность размещения таких курсов, автопроверки решений по задачам.

Методы исследования: анализ, синтез, наблюдение, сравнение. Источник данных - документация формата задач, статьи авторов, платформы выполняющие подобную задачу.

В ходе работы было разработано и внедрено в сервис Ulearn в компании Скб Контур решение позволяющее поддерживать задачи в формате Codeforces.

\end{abstract}

%%% Local Variables: 
%%% mode: latex
%%% TeX-master: "rpz"
%%% End: 
