\definecolor{gray}{rgb}{0.4,0.4,0.4}
\definecolor{darkblue}{rgb}{0.0,0.0,0.6}
\definecolor{cyan}{rgb}{0.0,0.6,0.6}
\lstloadlanguages{C,C++,csh,Java}

\lstset{
  basicstyle=\ttfamily,
  columns=fullflexible,
  showstringspaces=false,
  commentstyle=\color{gray}\upshape
}

\lstdefinelanguage{XML}
{
  morestring=[b]",
  morestring=[s]{>}{<},
  morecomment=[s]{<?}{?>},
  stringstyle=\color{black},
  identifierstyle=\color{darkblue},
  keywordstyle=\color{cyan},
  morekeywords={xmlns,version,type}% list your attributes here
}

\section{Инфраструктура Ulearn}
\label{cha:design}


В этом разделе описана текущая инфраструктура в образовательной системе Ulearn, а именно
\begin{enumerate}
\item как создаются слайды курса;
\item как происходит их обработка;
\item как осуществляется проверка отправленного студентом решения.
\end{enumerate}


\subsection{Структура курса}
Курс в системе Ulearn состоит из множества слайдов, каждый из которых может быть одним из трех типов:
\begin{enumerate}
\item Lesson — теоретический слайд-урок;
\item Quiz — тест с вопросами разных типов;
\item Exercise — задача на программирование.
\end{enumerate}

Задачи на программирование делятся на 3 типа: 
\begin{enumerate}
\item SingleFileExercise — задача на языке C\# из одного файла, где проверяется, что вывод равен константе;
\item CsProjectExercise — задача на языке C\#;
\item UniversalExercise — задача на языке, который определен конфигурацией docker-образа под эту задачу.
\end{enumerate}

Каждому слайду соответствует xml-файл, который задает тип слайда, информацию, которая будет показана при его отображении и прочие данные. 

Типизация объектов в xml-файле задаётся в соответствующей xsd-схеме.

Рассмотрим несколько конфигурации нескольких типов слайдов.
Первый тип представлен в листинге 1.4.

\begin{lstlisting}[caption={Пример xml-файла lesson-слайда}, language={XML}]
<?xml version='1.0' encoding='UTF-8'?>
<slide xmlns="https://ulearn.me/schema/v2" title="Theory" id="7ac1f6d9-41d6-4309-8eda-8786ccc3f990">
    <markdown>
        // Slide data
    </markdown>
</slide>
\end{lstlisting}

В данной xml-разметке находится главный тег — slide у которого присутствуют обязательные атрибуты: title (название данного слайда) и id (уникальный идентификатор слайда). Внутри тега slide находится тег markdown в который можно размещаться всю информационную нагрузку слайда: текст, ссылки и тд.
\begin{lstlisting}[caption={Пример xml-файла exercise-слайда}, language={XML}]
<?xml version="1.0"?>
<slide.exercise id="717f67cc-f53f-45e3-92f1-8fa20572c6d5" title="Practice" xmlns="https://ulearn.me/schema/v2">
  <scoring group="exercise" passedTestsScore="2" codeReviewScore="8"/>
  <markdown>
    // Task information
  </markdown>
  <exercise.universal exerciseDirName="src/01_fahrenheit" noStudentZip="true">
    <hideSolutions>true</hideSolutions>
    <hideExpectedOutput>true</hideExpectedOutput>
    <userCodeFile>task.js</userCodeFile>
    <includePathForChecker>../TestsRunner</includePathForChecker>
    <dockerImageName>js-sandbox</dockerImageName>
    <run>node docker-test-runner.js</run>
    <region>Task</region>
  </exercise.universal>
</slide.exercise>
\end{lstlisting}

В разметке слайда типа exercise, который представлен в листинге 1.5, различия начинаются с главного тега: вместо slide в данном типе находится slide.exercise. \\ Внутри slide.exercise, помимо уже известного markdown, присутствуют блоки: scoring — тег, отвечающий за конфигурацию расчета баллов для этой задачи, exercise.universal — тег, вместе со своими аттрибутами и вложенными тегами, отвечают за конфигурацию самого задания, механизма проверки и тд.

\subsection{Обработка конфигурации слайдов}

За обработку конфигураций слайдов ответственны сущности реализующие следующий контракт, представленный на листинге 1.6:

\begin{lstlisting}[caption={Контракт конфигуратора слайдов}]
public interface ISlide
	{
		Guid Id { get; }
		string Title { get; }
		SlideBlock[] Blocks { get; }
		
		void Validate(SlideLoadingContext context);
		void BuildUp(SlideLoadingContext context);
	}
\end{lstlisting}

Метод Validate отвечают за проверку корректности заполнения конфигурации.
А BuildUp — за логику построения по данным конфигурации html-разметки.


\subsection{Обработка конфигурации блоков}
За обработку конфигураций блоков внутри слайда ответственны сущности, которые наследуются абстрактного класса SlideBlock. Главные пункты этого класса соответствуют конфигураторам слайдов: методы Validate и BuildUp.

\subsection{Проверка решения студента}

После того как пользователь ввел своё решение в редактор сервиса и нажал кнопку "Отправить"     вызывается метод RunSolution контроллера ExerciseController, в который передаётся информация о курсе, слайде и отправленном решении. Его структура описана в листинге 1.7.

\begin{lstlisting}[language={C}, caption={Сигнатура метода RunSolution}]
public async Task<ActionResult<RunSolutionResponse>> RunSolution(
			[FromRoute] Course course, 
			[FromRoute] Guid slideId,
			[FromBody] RunSolutionParameters parameters,
			[FromQuery] Language language)
			
\end{lstlisting}

Данный метод осуществляет некоторые первичные проверки отправленного кода: на размер, на то, что код этой задачи уже проверяется у текущего пользователя и тд. А после этого ставит ответ студента в очередь на проверку.

Сервис RunCheckerJob с определенной частотой проверяет очередь отправленных решений. Если очередь непустая, то сервис забирает оттуда решение и запускает для него проверку в docker-контейнере. Во время запуска отправляет внутрь контейнера всё необходимое для проверки задачи: чеккер, решение студента и прочие файлы указанные при настройке задачи в xml-файле.

В Листинге 1.8 представлен пример конфигурирования docker-образа. 
Стоит обратить внимание на создание внутри образа нового пользователя student. Данный подход пригодится в дальнейшем, когда встанет задача конфигурирования образа и важно будет учесть права на чтения тестовых данных в режиме студента. 
\begin{lstlisting}[caption={Dockerfile для проверки задач по курсу JavaScript}]
FROM node:14-alpine

RUN apk update && apk upgrade

# Add new user for sandbox
RUN addgroup -S student && adduser -S -g student student \
    && chown -R student:student /home/student

COPY ./app/ /app/

RUN mkdir -p /source \
    && chown -R student:student /app

WORKDIR app

# Run user as non privileged.
USER student

# Install deps for tests.
RUN npm ci
\end{lstlisting}

Стоит заметить, что все чеккеры, необходимые для проверки решения студента должны быть написаны самим составителем курса и они будут специфичны для конкретного языка из-за зависимости от фреймворка тестирования.

После проверки из docker-контейнера возвращается результат в RunCheckerJob по которому мы можем узнать статус решения.

Бывают статусы, представленные в листинге 1.9:

\begin{lstlisting}[caption={Модель описывающая статус решения}]
public enum Verdict
	{
		Ok = 1,
		CompilationError = 2,
		RuntimeError = 3,
		SecurityException = 4,
		SandboxError = 5,
		OutputLimit = 6,
		TimeLimit = 7,
		MemoryLimit = 8,
		WrongAnswer = 9
	}
\end{lstlisting}

Результат сохраняется в базу данных и отображается пользователю.			
			
% \paragraph{Проверка} параграфа. Вроде работает.
% \paragraph{Вторая проверка} параграфа. Опять работает.

% Вот.

% \begin{itemize}
% \item Это список с <<палочками>>.
% \item Хотя он и не по ГОСТ, кажется.
% \end{itemize}

% \begin{enumerate}
% \item Поэтому для списка, начинающегося с заглавной буквы, лучше список с цифрами.
% \end{enumerate}

% Формула \ref{F:F1} совершено бессмысленна.

% %Кстати, при каких-то условиях <<удавалось>> получить двойный скобки вокруг номеров формул. Вопрос исследуется.

% \begin{equation}
% a= cb
% \label{F:F1}
% \end{equation}


% Окружение \texttt{cases} опять работает (см. \ref{F:F2}), спасибо И. Короткову за исправления..


% \begin{equation}
% a= \begin{cases}
%  3x + 5y + z, \mbox{если хорошо} \\
%  7x - 2y + 4z, \mbox{если плохо}\\
%  -6x + 3y + 2z, \mbox{если совсем плохо}
% \end{cases}
% \label{F:F2}
% \end{equation}

% \section{Подсистема всякой ерунды}

% Культурная вставка dot-файлов через утилиту dot2tex (рис.~\ref{fig:fig02}).

% \begin{figure}
%   \centering
% % [width=0.5\textwidth] --- регулировка ширины картинки
%   \includegraphics{figures/pic01}
%   \caption{Рисунок}
%   \label{fig:fig02}
% \end{figure}


% \subsection{Блок-схема всякой ерунды}

% \subsubsection*{Кстати о заголовках}

% У нас есть и \Code{subsubsection}. Только лучше её не нумеровать.

%%% Local Variables:
%%% mode: latex
%%% TeX-master: "rpz"
%%% End:
